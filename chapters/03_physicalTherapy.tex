\chapter{Physical Therapy for Senior Citizens}
This chapter presents the physical activity and health situation in Norway and internationally, and presents the Norwegian recommendation for physical activity. It then covers physical therapy in Norway and looks at how sensor technology can help improve their work.

\section{Physical activity and health}
In 2010 one of the largest ever systematic efforts to describe the global health situation was conducted. The article was later published in The Lancet~\cite{globalBurden}. One of their many findings was that since 1970 men and women have gained an additional ten years to their life expectancy, but spend more time living with injuries and illness. 

The amount of time adults spend in a sedentary position has increased over the last 30 years. The reasons for this are many, but increased use of technology and ease of transport are one of the main factors~\cite{sedentaryBehaviour}. An American study shows that one of four US adults spend 70\% of their waking hours in a sedentary position, 30\% in light activity, and little to no time is spent exercising. During the last decade research has started to emerge that links extended periods of sedentary time to metabolic risks~\cite{sedentaryTime}, obesity, and abnormal glucose metabolism~\cite{breaksSedentary}. It is suggested that prolonged periods of sedentary time should be avoided by increasing the number of breaks with during sedentary time. 

Based on these findings health recommendations regarding breaks in sedentary time should be added to the already existing ones~\cite{breaksSedentary}. Another study goes as far as stating that prolonged sedentary time is strongly related to metabolic risks independent of physical activity~\cite{sedentaryActivity}, and that elderly benefit more from reducing time spent sedentary than younger people. Though researchers agree that long periods of sedentary time is unhealthy, research still has to be done on how long a subject can stay sedentary before it has negative impact on the individuals health. How long one needs to stay active between periods of sedentary time, is also debated.

The \gls{ndh} has issued recommendations and guidelines pertaining to the minimum amount of physical activity for an elderly person. The recommendation is set to a minimum of 30 minutes a day with moderate activity~\cite{helsedirektoratetFysiskAktivitet}. In addition to this an elderly person should be standing in a skeleton bearing position for total of five hours per day to preserve the skeletons strength and form. Skeleton bearing position means that the subject is standing upright without any aid, such as a walking stick.

\section{Physical Therapy in Norway}
Physical therapy consists of two main steps: diagnosis and treatment. In the diagnosis phase the physiotherapist tries to determine what is wrong with the patient in order to apply the appropriate treatment. In case of elderly patients their problems might often be caused by lack of activity. When the therapist is finished with the diagnosis, a treatment plan is created. The plan often includes a set of exercises the patient should perform throughout the week, these are designed to let the patient reach his goal regarding incerased activity.

The treatment phase consists of the patients following the agreed upon plan to improve their activity level, or regain normal movement after operations or fractures. Some time after the patient has been presented with the treatment plan, the physiotherapist will return to the patient to monitor his progress. Often the patient may lack the dedication or motivation needed to follow the plan strictly. In such cases the plan might need to be revised or the physiotherapist needs to perform checkups more often to ensure that patient stays motivated.

\section{Sensor Technology in Physical Therapy}
There is no use of sensor technology to track activity in Norway today. Some of the physiotherapists we have been in contact with worked on research projects where such technology was utilized, but they had never used it outside of academic work.

Even though there is little use of this type of technology in physical therapy today, the Norwegian government has an increasing focus on the use of welfare technology. A committee formed by the Norwegian government called Hagen Utvalget (HU)~\cite{haagen} also concluded that there is a need for more welfare technology to tackle the ever increasing number of elderly. A system that could increase the effectiveness and quality of physiotherapist's work toward increasing the activity of the elder population, fits well with the technological goals presented in the \gls{hu}-report.  

\section{Accelerometer data in physical therapy}
Using accelerometer data to classify activity can help physiotherapist's both in the diagnosis and treatment phase. Having quantitative data of the users activity over one or more weeks, will give physiotherapists a much better overview of the patients current activity level. This can be helpful when creating a treatment plan.

Letting the patient see their improvement using information visualization can be a powerful tool for motivating the patient. Sometimes the improvement might be subtle and it can be hard for the patient to be motivated to continue the exercise plan without seeing some kind of indicator that they are in fact improving. Visualizing quantitative data can show these subtle improvements in the patients activity level, and display them in a way that motivates the patient to continue with the plan.  

\section{Certification of Medical Equipment}
\label{sec:medicalEquipment}
In 2006 a regulation concerning medical equipment in Norway came into effect~\cite{medicalEquipment}. The purpose of the regulation is to ensure that medical equipment used in Norway does not present a danger to either patient or users. To insure this, the regulation instils a set of requirements to both the use and the production of the equipment. In the regulations definition of medical equipment, standalone software is also to be perceived as medical equipment. This means that the system created in this project would need to be certified before such software could be used in practice. 
