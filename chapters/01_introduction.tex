\chapter{Introduction}

\section{Background}

The elderly population is in a constant rise, currently 15\% of the Norwegian population is above the age of 65. By the turn of the century the elderly population is expected to double~\cite{elder}. The initial rise of the elder population will be most noticeable in those below the age of 80. After 2025 a great increase in the population above the age of 80 is expected. The \gls{niph} reports that two out of three above the age of 75 consider themselves having ``good health'' but only a third preserve this level of health until death~\cite{elder}. The amount of time adults spend in a sedentary position has increased over the last 30 years. The reasons for this are many, but increased use of technology and ease of transport are one of the main factors~\cite{sedentaryBehaviour}.

With such predictions the Norwegian government formed a committee known as \gls{hu} to investigate the current situation and suggest solutions for accommodating the increase in the percentage of elderly~\cite{haagen}. One of the conclusions in the report was that too little of today's technology is incorporated as welfare technology for the elderly. A Danish report refereed to by \gls{hu} states that around 20\% of the tasks performed by healthcare personnel could be completely or partly replaced by technology~\cite{kmd}. 

Recent studies on the acitivy level of aduls and elderly in Norway show that only 1 in 5 reach the national goal of 30 minuites of acitivty each day~\cite{fysiskAktivitet2009}. Increasing the overall acitivty level of elderly is one of the main foci in the \gls{hu} report. To handle the rising percentage of elderly with sedentary behaviour, \gls{hu} suggests a national three step program that focuses on using welfare technology to diminish falls, social isolation and cognitive failure. Thus improving the overall quality of life for the elder population as well as reducing the workload for health care personnel. Step 3 states that:
\begin{quote}
\textit{Opt on technology that stimulates, activates and structures daily life.}
\end{quote}

We wish to address step 3 in the national program through the use of \gls{pi} technology. \gls{pi} technology is a set of tools that individuals can use to gather quantitative data about themselves for the purpose of self-reflection and self-monitoring. By collecting data from an activity monitor and visualizing user patterns,  we hope to bring awareness to their activity levels, and identify periods of long sedentary time. The data can in addition be utilized in consultation with health care personnel and rehabilitation facilities, to improve treatment and motivation of patients.

\section{Related work}
A large collaborative project known as FARSEEING is currently in progress. FARSEEING is collaboration between 10 partners in 5 EU countries and is funded by the European Commission. The aim is to create a thematic network that promotes the healthy and independent living for the elderly. FARSEEING wishes is to improve fall prediction and prevention, support older adults through technology, and use unique proactive opportunities to keep adults in their own environment. A practical approach is taken and the entire project is divided into work packages which combined expand the research, technological development and overall knowledge in this area \cite{farseeing}.

\gls{ntnu} is responsible for \gls{wp5} and the overall objective for it is to develop and validate feasible telemedicine service models for detection of accidental falls, fall risk assessment and exercise counselling \cite{wp5}. The service models should not be dependent on a specific technological platforms or sensor systems used to collect and disseminate data. The overall objective has been divided into three different domains. The second domain is partly relevant to the work performed in this thesis.
\begin{quote}
\textit{To develop a service that can demonstarte an exhcange of information between the older person and caregivers about fall risk, e.g. health-care personell are given information to be used for clinical decision making about the older person's fall risk and related movements.}
\end{quote}
%Syntes vi burde nevne det som står under et sted, men ikke nødvendigivs på slutten av denne seksjonen eller på denne måten.
Our paper and work has no direct relation to the FARSEEING project, but our advisor and the people we have been in contact with are a part of the project. This leads to a certain influence by the agenda of the FARSEEING project, and we hope that some of our work will aid them in the future.


\section{Research questions}
The main question we are attempting to answer is: \textit{Which visualizations are most fitting to aid physiotherapists in interpreting and understanding \gls{amu} data in communication with patients and other healthcare workers?}

In order to understand the problem, we have divided it into 3 Research Questions, each dealing with a specific problem. First we need to understand typical use cases in which a physiotherapist will use \gls{amu} data and how they are utilized. Talks with domain experts should allow us to develop basic requirements that are applicable to all visualizations that represent \gls{amu} data.

Research questions:
\vspace{-15pt}
\begin{description}[parsep=0pt, itemsep=0pt]
\item[Research Question 1:] What are the relevant use cases for visual presentation of the \gls{amu} data in physical therapy, from the physiotherapist's perspective?

\item[Research Question 2:] What are the basic requirements for visualizations of \gls{amu} data in scenarios identified by the physiotherapists?

\item[Research Question 3:] What are the preferred visualizations by the physiotherapists for the scenarios and requirements?
\end{description}

\section{Thesis outline}
