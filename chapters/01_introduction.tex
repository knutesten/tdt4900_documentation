\chapter{Introduction}

\section{Background}

The elderly population is in a constant rise, currently 15\% of the Norwegian population is above the age of 65. By the turn of the century the elderly population is expected to double \cite{elder}. The initial rise of the elder population will be most noticeable in those below the age of 80. After 2025 a great increase in the population above the age of 80 is expected. The \gls{niph} reports that two out of three above the age of 75 consider themselves having ``good health'' but only a third preserve this level of health until death \cite{elder}.

With such predictions the Norwegian government formed a committee known as \gls{hu} to investigate the current situation and suggest solutions for accommodating the increase in the percentage of elderly \cite{haagen}. One of the conclusions in the report was that too little of today's technology is incorporated as welfare technology for the elderly. A Danish report refereed to by \gls{hu} states that around 20\% of the tasks performed by healthcare personnel could be completely or partly replaced by technology \cite{kmd}.

The amount of time adults spend in a sedentary position has increased over the last 30 years. The reasons for this are many, but increased use of technology and ease of transport are one of the main factors \cite{sedentaryBehaviour}. An American study shows that 1 of 4 US adults spend 70\% of their waking hours in a sedentary position, 30\% in light activity, and little to no time is spent exercising.

To handle the rising percentage of elderly with sedentary behaviour, \gls{hu} suggests a national three step program that focuses on using welfare technology to diminish falls, social isolation and cognitive failure. Thus improving the overall quality of life for the elder population as well as reducing the workload for health care personnel. Step 3 states that:
\begin{quote}
\textit{``Opt on technology that stimulates, activates and structures daily life.''}
\end{quote}
% FIXED: REVIEW: We should probably write something around this. Now we just mention one of the steps without saying why. 
 
We wish to address step 3 in the national program through the use of \gls{pi} technology. \gls{pi} has become an emerging branch of technology to help individuals collect personal information for the purpose of self-reflection and self-monitoring. By collecting data from an activity monitor and visualizing user patterns we hope to bring awareness to their activity levels, and identify periods of long sedentary time. The data can in addition be utilized in consultation with health care personnel and rehabilitation facilities.

\section{Related work}
Farseeing articles, skulle få det av Svanæs?

\section{Research questions}
The main problem we are attempting to answer is: Which visualizations are most fitting to aid physiotherapists in interpreting and understanding \gls{amu} data?

(Should we explain the standard scenario here? Where they wear the \gls{amu} for a week, or is it too soon?)

In order to understand the problem, we have divided it into 3 Research Questions, each dealing with a specific problem. First we need to understand typical use cases in which a physiotherapist will use \gls{amu} data and how they are utilized. Talks with domain experts should allows us to develop basic requirements that are applicable to all visualizations that represent \gls{amu} data.

 A final set of visualizations based on the feedback will be created and rated.
% REVIEW: Should we use future tense in this section? Should we use future tense at all in our report except for in the future work-section?

\begin{description}
\item[Research Question 1:] What are the relevant use cases for visual presentation of the \gls{amu} data in physical therapy, from the physiotherapist's perspective?

\item[Research Question 2:] What are the basic requirements for visualizations of \gls{amu} data in scenarios identified by the physiotherapists?

\item[Research Question 3:] What are the preferred visualizations by the physiotherapists for the scenarios and requirements?
\end{description}

\section{Thesis outline}
