\chapter{Human-computer interaction and user centered design}

%Avsinttet var veldig tynt, starter ganske bra, men s� plutselig hoppper du til iterasjonsprossesen. Du nevner at en factor er � putte brukeren i sentrum, heter ikke dette user centered design? Hva slags andre metodolgier er det. Dette med design prinsippene (iterasjonene) er veldig fint � ha med men burde kanskje skrive litt mer om selveste feltet f�rst og design prinsippene. Det er greit � ha med siden vi tross alt har et kapittel som heter research design hvor vi skriver hvorfor vi akkurat valgte det vi gjorde. Du mangler kilder p� alt, motherfucking sources son!
\section{HCI Overview (working title)}
The science of \gls{hci} studies the interaction between humans and computers. The goal is to make this interaction as smooth and seamless as possible. Research done in the \gls{hci}-field have resulted in guidelines and design methods for the creation of efficient and usable software. One important factor when focusing on usability is to include the future users the design process to get feedback on the user interface.
% I have no sources on this, this is just common knowledge... Don't know where to find sources about it either.

\section{Information Visualization}
Information visualization is useful when you want to display large amounts of data simultaneously. The human brain is not good at getting an overview of the information by looking at large tables of numbers or text. Visualizations utilize the strengths of human cognition. By using computer graphics to make visualizations, large amounts of information can be displayed in a way that humans can process and analyse quickly and intuitively.

Many guidelines have been suggested for creation of the optimal visualization. Shneiderman summarized many of these principles in his \emph{visual-information-seeking mantra}~\cite{shneiderman}:

\vspace{-13pt}
\begin{quote}
\textit{Overview first, zoom and filter, then details on demand.}
\end{quote}

For the visualization to be effective the user has to get an overview of how the information is presented. The system should also allow for zooming in on interesting parts of the information as well as giving the opportunity to filter out information that is not of interest. Most visualizations remove some level of detail for the data to be more accessible and easier to read. Therefore it is important that the system allows the user to access more detailed data when needed.

Shneiderman also identifies the importance of showing the relationships between different items, allowing the user to extract subcollections of the displayed items and storing the user action history to allow for undo/redo functionality.

\subsection{Interactivity}
One of the benefits of creating visualizations for computers is the ability to add interactivity. Shneidermann mentions 6 tasks that should be implemented when creating information visualizations:
\vspace{-3mm}
\begin{description}[itemsep=0cm, parsep=0cm]
  \item[Overview] Show an overview of the entire data set
  \item[Zoom] Let the user look closer at elements of interest
  \item[Filter] The user should be able to filter out tasks that are not interesting
  \item[Details-on-demand] Show details when elements are selected
  \item[History] Let the user undo or redo actions
  \item[Extract] Allow the user to extract a subset of the entire data set
\end{description}

The first thing the users sees when interacting with an information visualization is an overview of the data set. This can be done by zooming out and then let the user zoom in on areas of interest. Another approach is to aggregate the data in separate sections that can then be investigated further.

It is rare that the user is equally interested in every part of the data set, therefore it is useful to be able to zoom in on elements for a more detailed look. It is important to make sure that the user does not loose their sense of position and get lost in the visualization.

Often the some of the data in the set is not relevant and only distorts the visualization. In such cases one should be able to filter out those elements. The user should be able to filter out unwanted data using sliders, buttons etc. The update should happen real-time, to allow the user to see how the filter affects the visualization. 

Typically information visualizations hide the numerical data (or other detailed data) behind the element. It is therefore paramount to give the user the ability to access this data when it is needed. The user should be able to select an element or a small group of elements and browse the details in a list or other textual representation.

When working with a visualizations where you can make changes to filter, zoom, etc. It is useful for the user to be able to undo and redo tasks. Undo will give the user the ability to go back form an undesired zoom level or filter setting quickly. Users will often times do make more actions than they can keep track of, so giving them the ability to trace their steps improves usability.

When the user has used the visualization and found what he was looking for, the user should be able to extract those elements. The extracted elements should be saved in a format that can be sent to and seen by others. 


\subsection{Visual Variables} 
One important factor to consider when creating a visualizations are which visual variables to use. To create an intuitive visualization it is important use the different visual variables correctly. We will briefly go through some of the more common visual variables here and discuss how and how not to use them, based on Carpendales article about the subject~\cite{carpendale}.

\textbf{Visual variables:}
\begin{table}[h!]
  \begin{tabular}{|l|p{10cm}|}
      \hline
      Position    & Position of object, for example x- and y-coordinates in a two dimensional system. \\ \hline
      Size        & Size of object. \\ \hline
      Value       & Change in colour scale from light to dark. \\ \hline
      Colour      & Change hue for given value, for example blue, red and yellow. \\ \hline
  \end{tabular}
  \caption{Overview of visual variables.}
\end{table}

\textbf{Characteristics of visual variables:}
\begin{table}[h!]
  \begin{tabular}{|l|p{10cm}|}
      \hline
      Selective   & Will a change in this variable make it easier to select it from a group of variables? \\ \hline
        Associative & Will changes in variables make us able to distinguish different groups of variables \\ \hline 
        Qualitative & Can the visual variable be used to illustrate the numerical value and relationship between variables? \\ \hline
        Order       & Will changes in the visual variable allow us to order them? \\ \hline
        Length      & How many changes is it possible to distinguish between for this type of variable? \\ \hline
    \end{tabular}
    \caption{Description of characteristics of visual variables}
\end{table}

Position fulfils all the characteristics of visual variables. When thinking of a scatter plot it is easy to see that positioning the points will make them selective and associative. By using scales, position can be used to show the value of the variable in a numerical sense, so it is also quantitative. Order is also fulfilled, a ruler is a simple example of how position can be used to order. The length is theoretically infinite, and only restricted by the screen resolution.

Size fulfils all of the characteristics. It is both selective and associative, humans can easily identify for example the smallest circle in a group of circles. Though size can be used to visualize a numerical value, it is often hard for humans to accurately see how much larger one object is compared to another. Different sizes can easily be ordered. As for the length of this variable Carpendale suggest about five different sizes for selection and about 20 different sizes for distinction. It is important to note that humans can identify small changes in size when objects are close, however when the distance between objects is increased it is hard to distinguish between these differences.

Value can be used both for selection and association. Humans can identify darker parts and group them with ease. It is not quantitative, it is difficult to identify that one tone of grey is twice as dark as another. One can however say that one grey tone is darker than another, and therefore you can order them. Carpendale suggest that the length of this variable is less than 7 for selection, and about 10 for distinction.

Colours are selective and associative. Unless you are colour blind you can easily identify and group different colours. Colours are not quantitative, it is hard for humans to say that one colours is twice that of another. Though we have colour scales, humans do not intuitively order colours, this can easily be illustrated by a question: Which colours is greater blue or yellow? Most people will not have an answer. Carpendale suggests that you use less than 7 different colours for selection, and about 10 for distinction.

\section{Prototyping}
A prototype is a realization of a design that stakeholders can interact with and explore. The limitation of a prototype is that it often only focuses on one product characteristic and neglects the others. A prototype can be anything from a complex piece of software to a simple storyboard or sketch. Prototypes serve as an aid by clarifying communication between team members, and efficiently exploring design ideas with stakeholders and designers. Building the prototype itself encourages reflection of the design and is recognized by designers from many disciplines as an important aspect of the design process \cite{interactionDesign}.

\subsection{Low-fidelity vs. High fidelity}
Low-fidelty prototypes do not resemble the finish product very much. It often uses completely different and much cheaper materials then the final product making them cheap, simple and easily modifiable. Examples of such prototypes are storyboards, sketching, and wizard of Oz. Low-fidelity are important in early development stages because the simplicity encourages exploration and modification. The disadvantage is that these prototypes are never kept or integrated into the final product.

High-fidelity prototypes are much closer to the final product, and therefore give a much stronger impression of the final product. The high-fidelity prototype is useful for identifying technical issues and selling ideas to people. These prototypes are often reused or developed into a final product, but require more time and resources to create.

The very nature of a prototype involves making compromises. Therefore the choice of prototype lies in what kind of of questions we want to answer. Two common compromises that are often traded against each other are breadth of functionality vs depth of functionality. \textit{Horizontal prototyping} focuses on a wide range of functions but little details and \textit{vertical prototyping} focuses on providing a lot of detail for a few functions. 

\section{Focus groups}
Focus group, or group interview is an informal technique that can help software developers identify the users needs and feelings about a system. The technique can be used both before interface design and long after the system has been implemented. 

According to Nielsen~\cite{focusGroup} the focus group should have at least six participant to maintain a flowing discussion and provide different perspectives. The participants should be representative of the intended users, or be the final users themselves. Sessions normally last two hours and are run by a moderator. The moderators job is to keep the discussion flowing and let everyone get their point across. Moderators can also guide the discussion in the direction relevant to the goals of the focus groups. 

A single session may not be representative enough or can get sidetracked, it is therefore important to run more than one focus group. If improvements suggested by participants have been implemented it is important to run another iteration of the focus group with the same participants to attain feedback on the changes that have been made. Enough focus groups have been conducted when new information is no longer being received, i.e. a point of saturation has been reached~\cite{howFocusGroup}. 

Nielsen~\cite{focusGroup} discusses two pitfalls with focus groups: Because sessions are in groups the users do not test the system themselves, instead they are presented with a demo. The problem with such a demo is that the participants never have to question what to do next or consider the meaning of the screen options. The second problem is that what participants say they want, is not necessarily what they need. Also, ideas described by the moderator might be perceived differently by the participants. By providing concrete examples through prototypes of the technology, one can minimize this problem.

\section{Validity (Should this be here?)}

