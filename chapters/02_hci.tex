\chapter{Human-computer interaction and user centered design}

\section{HCI Overview (working title)}

\section{Information Visualization}
Information visualization is useful when you want to display large amounts of data simultaneously. The human brain is not good at getting an overview of the information by looking at large tables of numbers or text. Visualizations utilizes the strengths of human cognition. Using computer graphics to make visualizations, large amounts of information can be displayed in a way that humans can process and analyse quickly and intuitively.

Many guidelines have been suggested for creation of the optimal visualization. Shneiderman summarized many of these principles in his \emph{visual-information-seeking mantra} \cite{shneiderman}:
%Seems like the space between this paragraph and the quote is greater than the quote in the introduction...
\begin{quote}
\textit{Overview first, zoom and filter, then details on demand.}
\end{quote}

For the visualization to be effective the user has to get an overview of how the information is presented. The system should also allow for zooming in on interesting parts of the information as well as giving the opportunity to filter out information that is not of interest. Most visualizations remove some level of detail for the data to be more accessible and easier to read. Therefore it is important that the system allows the user to access more detailed data when needed.

Shneiderman also identifies the importance of showing the relationships between different items, allowing the user to extract subcollections of the displayed items and storing the user action history to allow for undo/redo functionality.

\subsection{Visual Variables}
One important factor to consider when creating a visualizations are which visual variables to use. To create an intuitive visualization it is important use the different visual variables correctly. We will briefly go through some of the more common visual variables here and discuss how and how not to use them, based on Carpendales article about the subject \cite{carpendale}.

%This should be changed to a table.
Visual variables:
\begin{table}[h!]
    \begin{tabular}{|l|p{10cm}|}
        \hline
        Position    & Position of object, for example x- and y-coordinates in a two dimensional system. \\ \hline
        Size        & Size of object. \\ \hline
        Value       & Change in colour scale from light to dark. \\ \hline
        Colour      & Change hue for given value, for example blue, red and yellow. \\ \hline
    \end{tabular}
    \caption{Overview of visual variables.}
\end{table}

Characteristics of visual variables:
\begin{table}[h!]
    \begin{tabular}{|l|p{10cm}|}
        \hline
        Selective   & Will a change in this variable make it easier to select it from a group of variables? \\ \hline
        Associative & Will changes in variables make us able to distinguish different groups of variables \\ \hline 
        Qualitative & Can the visual variable be used to illustrate the numerical value and relationship between variables? \\ \hline
        Order       & Will changes in the visual variable allow us to order them? \\ \hline
        Length      & How many changes is it possible to distinguish between for this type of variable? \\ \hline
    \end{tabular}
    \caption{Description of characteristics of visual variables}
\end{table}

Position fulfils all the characteristics of visual variables. When thinking of a scatter plot it is easy to see that positioning the points will make them selective and associative. By using scales, position can be used to show the value of the variable in a numerical sense, so it is also quantitative. Order is also fulfilled, a ruler is a simple example of how position can be used to order. The length is theoretically infinite, and only restricted by the screen resolution.

Size fulfils all of the characteristics. It is both selective and associative, humans can easily identify for example the smallest circle in a group of circles. Though size can be used to visualize a numerical value, it is often hard for humans to accurately see how much large one object is compared to another. Different sizes can easily be ordered. As for the length of this variable Carpendale suggest about five different sizes for selection and about 20 different sizes for distinction. It is important to note that humans can identify small changes in size when objects are close, however when the distance between objects is increased it is hard to distinguish between these differences.

Value can be used both for selection and association. Humans can identify darker parts and group them with ease. It is not quantitative, it is difficult to identify that one tone of grey is twice as dark as another. One can however say that one grey tone is darker than another, and therefore you can order them. Carpendale suggest that the length of this variable is less than 7 for selection, and about 10 for distinction.

Colours are selective and associative. Unless you are colour blind you can easily identify and group different colours. Colours are not quantitative, it is hard for humans to say that one colours is twice that of another. Though we have colour scales, humans do not intuitively order colours, this can easily be illustrated by a question: Which colours is greater blue or yellow? Most people will not have an answer. Carpendale suggests that you use less than 7 different colours for selection, and about 10 for distinction.

\subsection{Interactivity}

\section{Prototyping}

\section{Focus groups}
Focus group, or group interview is an informal technique that can help software developers identify the users needs and feelings about the system in question. The technique can be used both before interface design and long after the system has been implemented. 

According to Nielsen \cite{focusGroup} the focus group should have at least six participant to maintain a flowing discussion and provide different perspectives. The participants should be representative of the intended users, or be the final users themselves. Sessions normally last two hours and are run by a moderator. The moderators job is to keep the discussion flowing and let everyone get their point across. Moderators can also guide the discussion in the direction relevant to the goals of the focus groups. 



A single session may not be representative enough or can get sidetracked, it is therefore important to run more than one focus group. 
%REVIEW: Though the above might be correct, I think it is more important to mention that several iterations are required to take the ideas of previous focus groups modify product and then get feedback on the modifications. Alternatively you can say that several focus groups are needed to get all the ideas out of the participants, because of several factors (there were some sections about this in the papers you forced me to read, you can refer to them).

Nielsen \cite{focusGroup} discusses two pitfalls with focus groups: Sessions are in groups therefore the participants are unable to test the system themselves, so a demo is normally provided instead. The problem with such a demo is that the participants never have to question what to do next or consider the meaning of the screen options. The second problem is that what participants say they want is not necessarily what they need, or things described by the moderator might be perceived differently by the participants. By providing concrete examples through prototypes of the technology, one can minimize this problem.
% Isn't having several iterations of the focus group also a way to combat the last problem?

\section{Validity (Should this be here?)}

