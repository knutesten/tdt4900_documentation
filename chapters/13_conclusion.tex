\chapter{Conclusion}
%\item[Research Question 2:] What are the functional and user experience requirements for visualizations of \gls{imu} data in scenarios identified by the physiotherapists?

\section{RQ1: User Scenarios}


\begin{enumerate}[itemsep=0cm, parsep=0cm]
\item When analysing patients activity level, either individually or in cooperation with occupational therapists or other physiotherapists.
\item In consultation with the patient.
\item In communication with nursing homes and home care personnel.
\item In consultation with next of kin.
\item For educational purposes.
\end{enumerate}


\section{RQ2: Requirements}
The requirements have had several iterations through the project. It started with the initial requirements outlined in chapter~\ref{ch:initialRequirements} which were iterated upon in focus groups 1 (chapter \ref{ch:focusGroup1}). In addition a set of \gls{ux} requirements were outline after focus group 1. Table~\ref{tab:f2ReqCon} and~\ref{tab:f2ReqUxCon} display what we consider to be the final functional and \gls{ux} requirements for visualizations of data identified by the physiotherapists.
\begin{table}[h!]
  \begin{center}
  \begin{tabular}{|c|p{12cm}|}
    \hline
      \textbf{Id} & \textbf{Requirement} \\ \hline
    \multicolumn{2}{|l|}{The visualizations should \ldots} \\ \hline
      F2-1 & give the user an overview of the week where the days are classified by national recommendations or personal goals \\ \hline
      F2-2 & show the activity level for each hour of the day \\ \hline
      F2-3 & make it easy to identify the length of activity intervals \\ \hline
      F2-4 & make it possible to compare multiple days \\ \hline
      F2-5 & make it easy to identify hours of the day where activity can be added \\ \hline
      F2-6 & show the activity level compared to national or personal goals \\ \hline
      F2-7 & let the user identify patients that are active during the night \\ \hline
      F2-8 & let you compare two separate weeks to see the patients progress \\ \hline
      F2-9 & should be printable in grayscale \\ \hline
      F2-10 & show the activity distribution for a day (sedentary, standing, walking) \\ \hline
      F2-11 & allow the users to toggle if nighttime should be included or not \\ \hline
  \end{tabular}
  \end{center}
  \caption{Functional requirements from the second focus group}
  \label{tab:f2ReqCon}
\end{table}

Both functional and UX requirements presented here should be used as a basis for creating visualizations in physiotherapy. We assume that these requirements will be expanded upon when developing a system for a specific customer and purpose.

\begin{table}[h!]
  \begin{center}
  \begin{tabular}{|c|p{12cm}|}
    \hline
      \textbf{Id} & \textbf{Requirement} \\ \hline
    \multicolumn{2}{|l|}{The visualizations should \ldots} \\ \hline
      F2-11 & not be judgemental towards the patients activity level \\ \hline
      F2-12 & should be honest about the patients activity level \\ \hline
      F2-13 & should motivate the patient to be more active \\ \hline
      F2-14 & should be intuitive and easy to understand for the user and third parties \\ \hline
      F2-15 & be easy to explain to cognitively capable patients \\ \hline
  \end{tabular}
  \end{center}
  \caption{User experience requirements from the second focus group}
  \label{tab:f2ReqUxCon}
\end{table}

\section{Further work}


