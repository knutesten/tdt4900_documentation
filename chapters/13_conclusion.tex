\chapter{Conclusion}



\section{Conclusion}
During the project we conducted two focus groups and created two prototype systems. These form the basis for most of our results. The first research question asked what types of scenarios physiotherapists saw as relevant when using visualizations to display qualitative data from sensors such as the activPAL. To answer this question we created a set of initial scenarios by talking to a domain expert at St. Olav's Hospital. These initial scenarios were then discussed with the participants from the two consecutive focus groups to create five scenarios that all the physiotherapists agreed on:
\begin{table}[h!]
  \begin{tabular}{|c|p{10cm}|}
    \hline
    \textbf{Id} & \textbf{Scenario} \\ \hline
    S1 & When analysing patients activity level, either individually or in cooperation with occupational therapists or other physiotherapists. \\ \hline
    S2 & In consultation with the patient. \\ \hline
    S3 & In communication with nursing homes and home care personnel. \\ \hline
    S4 & In consultation with next of kin. \\ \hline
    S5 & For educational purposes. \\ \hline
  \end{tabular}
\end{table}

The second research question was concerned with the functional and \gls{ux} requirements for visualizations of data from sensors such as activPAL. The initial requirements were created after an interview with a domain expert at St. Olav's Hospital. The requirements were then discussed and revised in the two focus groups. The result after the two revision can be seen below:

\begin{table}[h!]
  \begin{center}
  \begin{tabular}{|c|p{12cm}|}
    \hline
      \textbf{Id} & \textbf{Requirement} \\ \hline
    \multicolumn{2}{|l|}{The visualizations should \ldots} \\ \hline
      F2-1 & give the user an overview of the week where the days are classified by national recommendations or personal goals \\ \hline
      F2-2 & show the activity level for each hour of the day \\ \hline
      F2-3 & make it easy to identify the length of activity intervals \\ \hline
      F2-4 & make it possible to compare multiple days \\ \hline
      F2-5 & make it easy to identify hours of the day where activity can be added \\ \hline
      F2-6 & show the activity level compared to national or personal goals \\ \hline
      F2-7 & let the user identify patients that are active during the night \\ \hline
      F2-8 & let you compare two separate weeks to see the patients progress \\ \hline
      F2-9 & should be printable in grayscale \\ \hline
      F2-10 & show the activity distribution for a day (sedentary, standing, walking) \\ \hline
      F2-11 & allow the users to toggle if nighttime should be included or not \\ \hline
  \end{tabular}
  \end{center}
  \caption{Functional requirements from the second focus group}
  \label{tab:f2ReqCon}
\end{table}

\begin{table}[h!]
  \begin{center}
  \begin{tabular}{|c|p{12cm}|}
    \hline
      \textbf{Id} & \textbf{Requirement} \\ \hline
    \multicolumn{2}{|l|}{The visualizations should \ldots} \\ \hline
      F2-11 & not be judgemental towards the patients activity level \\ \hline
      F2-12 & should be honest about the patients activity level \\ \hline
      F2-13 & should motivate the patient to be more active \\ \hline
      F2-14 & should be intuitive and easy to understand for the user and third parties \\ \hline
      F2-15 & be easy to explain to cognitively capable patients \\ \hline
  \end{tabular}
  \end{center}
  \caption{User experience requirements from the second focus group}
  \label{tab:f2ReqUxCon}
\end{table}

The third research question asked what types of visualizations would be preferred by the physiotherapists for the scenarios and requirements stated above. Two set of prototypes were created to answer this question. The first prototype had a total of nine different visualizations, whereof four were discarded after the first focus group. For the second focus group a new prototype was created using the feedback from the first focus group. The second prototype contained six visualizations, one was discarded after the second focus group. Time did not permit to create a new prototype using feedback from the second focus group, so not all of the requirements were covered by the visualizations. The below tables shows which visualizations satisfies which requirements and scenarios:

\begin{table}[h!]
  \centering
  \begin{tabular}{|c|c|c|c|c|c|c|c|c|c|c|c|}
    \hline
    & F2-1 & F2-2 & F2-3 & F2-4 & F2-5 & F2-6 & F2-7 & F2-8 & F2-9 & F2-10 & F2-11 \\ \hline
    U1 & + & -- & -- & + & -- & -- & -- & -- & + & -- & -- \\ \hline
    F1 & -- & -- & -- & + & -- & -- & -- & -- & + & + & -- \\ \hline
    F3 & -- & -- & + & -- & + & -- & -- & -- & -- & + & -- \\ \hline
    T1 & -- & + & -- & + & + & + & + & -- & -- & -- & -- \\ \hline
    T5 & -- & + & -- & + & + & + & + & -- & + & -- & -- \\ \hline
  \end{tabular}
  \caption{Table showing which requirements each visualization satisfies.}
  \label{tab:reqSat}
\end{table} 

\begin{table}[h!]
  \centering
  \begin{tabular}{|c|c|c|c|c|c|}
    \hline
       & S1 & S2 & S3 & S4 & S5 \\ \hline
    U1 & +  & +  & -- & +  & +  \\ \hline
    F1 & +  & +  & +  & +  & +  \\ \hline
    F3 & +  & -- & +  & -- & +  \\ \hline
    T1 & +  & -- & +  & -- & +  \\ \hline
    T5 & +  & +  & +  & +  & +  \\ \hline
  \end{tabular}
  \caption{Table showing which scenarios each visualization satisfies.}
  \label{tab:scenSet}
\end{table}

\section{Further work}
We have merely scratched the surface in terms of the potential that visualizations have to help users understand sensor data in an interesting and simple manner. Below are some of the authors suggestions on possible areas that can be investigated further:

\begin{itemize}
  \item Further improve the  prototype to implement the final requirements and developing it into a full application ready for user and patient interaction.
  \item Conduct focus groups, usability tests, or interviews to receive feedback from potential patients, and investigate how they perceive and react to visualizations based on personal sensor data.
  \item A field study would be of great interest, seeing how a prototype can be used in their daily work with patients. The focus group participants were eager to try it in the field, and the findings might greatly influence future requirements and guidelines.
  \item Investigate how available screen space will affect the visualizations and presentation. Is it possible to create an application that will be user friendly on smaller systems, such as tablets and smartphones.
  \item Investigate the effect of using qualitative data and information visualizations to motivate patients to more activity.
\end{itemize}

