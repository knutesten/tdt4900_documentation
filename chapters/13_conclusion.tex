\chapter{Conclusion}

% REVIEW: I agree that we can combine the sections so they don't seem so empty.
\section{RQ1: User Scenarios}
% REVIEW: Not sure if this is what we should have here.
We initially started with two scenario, but during the first focus group these quickly evolved into multiple scenarios, these are described in the table below:

\begin{table}[h!]
  \begin{tabular}{|c|p{10cm}|}
    \hline
    \textbf{Id} & \textbf{Scenario} \\ \hline
    S1 & When analysing patients activity level, either individually or in cooperation with occupational therapists or other physiotherapists. \\ \hline
    S2 & In consultation with the patient. \\ \hline
    S3 & In communication with nursing homes and home care personnel. \\ \hline
    S4 & In consultation with next of kin. \\ \hline
    S5 & For educational purposes. \\ \hline
  \end{tabular}
\end{table}

These scenarios are based on what kind of uses the physiotherapists imagined for the system after seeing it for the first time, and we had little say in the matter.

\section{RQ2: Requirements}
% REVIEW: Don't think we should talk about how we progressed here, just present our results. 
The requirements have had several iterations through the project. It started with the initial requirements outlined in chapter~\ref{ch:initialRequirements} which were iterated upon in focus groups 1 (chapter \ref{ch:focusGroup1}). In addition a set of \gls{ux} requirements were outline after focus group 1. Table~\ref{tab:f2ReqCon} and~\ref{tab:f2ReqUxCon} display what we consider to be the final functional and \gls{ux} requirements for visualizations of data identified by the physiotherapists.
\begin{table}[h!]
  \begin{center}
  \begin{tabular}{|c|p{12cm}|}
    \hline
      \textbf{Id} & \textbf{Requirement} \\ \hline
    \multicolumn{2}{|l|}{The visualizations should \ldots} \\ \hline
      F2-1 & give the user an overview of the week where the days are classified by national recommendations or personal goals \\ \hline
      F2-2 & show the activity level for each hour of the day \\ \hline
      F2-3 & make it easy to identify the length of activity intervals \\ \hline
      F2-4 & make it possible to compare multiple days \\ \hline
      F2-5 & make it easy to identify hours of the day where activity can be added \\ \hline
      F2-6 & show the activity level compared to national or personal goals \\ \hline
      F2-7 & let the user identify patients that are active during the night \\ \hline
      F2-8 & let you compare two separate weeks to see the patients progress \\ \hline
      F2-9 & should be printable in grayscale \\ \hline
      F2-10 & show the activity distribution for a day (sedentary, standing, walking) \\ \hline
      F2-11 & allow the users to toggle if nighttime should be included or not \\ \hline
  \end{tabular}
  \end{center}
  \caption{Functional requirements from the second focus group}
  \label{tab:f2ReqCon}
\end{table}

Both functional and UX requirements presented here should be used as a basis for creating visualizations in physiotherapy. We assume that these requirements will be expanded upon when developing a system for a specific customer and purpose.

\begin{table}[h!]
  \begin{center}
  \begin{tabular}{|c|p{12cm}|}
    \hline
      \textbf{Id} & \textbf{Requirement} \\ \hline
    \multicolumn{2}{|l|}{The visualizations should \ldots} \\ \hline
      F2-11 & not be judgemental towards the patients activity level \\ \hline
      F2-12 & should be honest about the patients activity level \\ \hline
      F2-13 & should motivate the patient to be more active \\ \hline
      F2-14 & should be intuitive and easy to understand for the user and third parties \\ \hline
      F2-15 & be easy to explain to cognitively capable patients \\ \hline
  \end{tabular}
  \end{center}
  \caption{User experience requirements from the second focus group}
  \label{tab:f2ReqUxCon}
\end{table}

\section{Further work}
We have merely scratched the surface in terms of the potential that visualizations have to help users understand sensor data in an interesting and simple manner. Below are some of the authors suggestions on possible areas that can be investigated further:

\begin{itemize}
  \item Further improve the  prototype to implement the final requirements and developing it into a full application ready for user and patient interaction.
  \item Conduct focus groups, usability tests, or interviews to receive feedback from potential patients, and investigate how they perceive and react to visualizations based on personal sensor data.
  \item A field study would be of great interest, seeing how a prototype can be used in their daily work with patients. The focus group participants were eager to try it in the field, and the findings might greatly influence future requirements and guidelines.
  \item Investigate how available screen space will affect the visualizations and presentation. Is it possible to create an application that will be user friendly on smaller systems, such as tablets and smartphones.
  \item Investigate the effect of using qualitative data and information visualizations to motivate patients to more activity.
\end{itemize}

