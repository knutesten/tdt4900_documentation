\chapter{Focus group 2}
\label{ch:focusGroup2}

\section{Procedure}

\subsection{Participants}
Participants from the previous focus group session were invited, but due to conflicts in schedule only 4 out of the 5 could make it. For this session there were three females and one male present. The ages were between 34--36.

\subsection{Location and procedure}
The focus group was conducted at St. Olav's Hospital, but we were unable to obtain the exact same room.  Much like the first session the participants were seated around a table facing a projector canvas. The visualizations were controlled by the assistant moderator while the primary moderator was responsible for conducting the session. Like the previous session it was video recorded and lasted for 2 hours.

\begin{enumerate}[itemsep=0cm, parsep=0cm]
  \item Introduction and questionnaire
  \item Discuss scenarios
  \item Review of visualizations
  \item Review the requirements
  \item Discussion
  \item Interview about physiotherapy in practice
\end{enumerate}

The introduction phase was brief, since we were already acquainted with the participants. After the agenda was presented the participants were asked to fill out a questionnaire about their familiarity with technological aids such as smartphones and tablets, as well as how they felt about using information visualizations in their work. 

After the questionnaire, scenarios created after the first focus group were presented to the participants so that feedback could be received. They were asked to provide additional scenarios if they felt that the existing ones did not cover all the possible uses of the technology presented.

The visualizations were presented one by one as in the first focus group. The participants then gave feedback on the modifications made as well as the new features. When all the visualizations had been presented and discussed the participants were given different colour choices for visualization T1. Laptops were used for this purpose as it was hard to differentiate between the colour gradients on the projector. A printed black and white version of visualizations T1 and T5 were provided to the participants, while the rest were shown on the computer in a grayscale.

Requirements created based on the feedback received from the first focus group were then presented to the participants. After a brief introduction to software requirements, the participants were asked to go through them and comment if they felt something was missing or unclear.

After the requirements had been reviewed the findings of the focus group were discussed so that there was no ambiguity. The last part of the focus group was used for an unstructured interview with the physiotherapists about how physiotherapy is practised in Norway. 

\section{Results}
% What should I say here my friend?

\subsection{Visualizations}
The second prototype had contained six visualizations, where five of them were modified versions of visualizations present in the first prototype and one was new. Other features that were added were the ability to set goals and a small chart in the form of circles for displaying how far the patient was from reaching his or her goal.

\subsubsection{Overview chart}
U1 was the only overview chart left after changes were made for the second prototype. Participants of the focus group were pleased with the fact that the smilies had been replaced with coloured squares. The new way of classifying days based on how close they were to their set goals was well received by all the participants. One of the participants stated that: ``It is good that we can set the goals ourselves. How active different patients should be differs greatly.''. Another participants suggested adding the current goal to the visualization to make it easier to determine what each classification means. Most of the participants liked the simplicity of U1 and said that this was a type of diagram that many of their patients would understand, however one of the participants felt that the lack of detail in the diagram made it useless in practical situations.

Most of the participants like the new overview chart. The currently set goal should be displayed so it is clear how the what the classification means.

\subsubsection{Aggregated charts}
F2, the box diagram, was discarded in the second prototype, but the illustrations used to show each type of activity was added to F1. Nighttime was also removed, as requested in the first focus group. The participants were satisfied with the changes and liked the fact that the illustrations were now added to the pie chart, F1. All the participants agreed that the F1 was a good way to see the overall activity of the day, and that excluding nighttime increased the practical use of this chart.

F3 was not changed other than the colour for sedentary activity being white instead of red. The participants did not like the colour change particularly, and felt that the red would be better than white. All the participants agreed that this was a good way to visualize interval length. Some of the participants suggested that you should have the ability toggle nighttime on and off. They stated that nighttime is relevant for some patients that are active during the night, but for most patients displaying nighttime would just be in the way.

Both F1 and F3 were well received by the physiotherapists. F3 should have functionality for toggling nighttime on and off.

\subsubsection{Timeline and Clock}
T2 and T3 were discarded after focus group 1, because the participants felt the amount of detail was unnecessary, as well as the active periods being hard to identify. T1 was the favourite visualization from focus group 1 and was not changed much for the second prototype. Goal circle diagrams were added to each day to display how the patient's activity level compared to the goal set. The participants did not like the addition of the goal circles. One suggested being able to toggle the goal circles on and off, or be able to look at the goal circles as a separate visualization. As in U1 the participants felt that the current value of the goal should be displayed so that you knew what the circles represented.

T4 was not particularly well received in the first focus group, but it was added to the second prototype with a colour change of the sedentary activity from red to white. Even with the colour change the participants all agreed that this type of graph was too detailed and was not useful in practice. The visualization should be discarded.

T5 was the only new visualization added for prototype 2, if not counting the goal circles. Though this visualization was made on the request of the participants, they were not overly enthusiastic about T5. Most of the participants preferred T1 to T5. However when asked if they would like the option of having both visualizations available they all agreed that T5 could be useful in some cases and should be kept. Also here the participants felt that the goal circles made the chart overly complex, and that this should be a toggle option.

T4 should be removed as it has no practical use. T1 and T5 should be kept, but the user should be able to toggle the goal circles on and off. The current goal should be shown when the goal circles are displayed. 

\subsection{Colours and printouts}
Different colour choices were discussed with the participants. Because the projector did not handle gradients well, a laptop was used to show different colour choices. For the gradient colour the participants preferred white to blue to black, and white to black. For representing the different activity types (sitting, standing, walking) the participants liked red, yellow and green. One participant expressed some concern with the fact that red, yellow and green was used for sitting, standing and walking for F1, F3 and T5, but in U1 it was used for classifications. Using the same colours with different meanings can be confusing. An alternative to the coloured boxes in U1 should be found.

Because physiotherapists working for Trondheim Kommune do not have access to a colour printer, grayscale printouts were showed to the participants. The participants found it hard to use T1 effectively when printed, because the printer did not handle grayscale gradients well. In this situation the participants preferred T5 over T1 since bars were used instead of colour for displaying the activity level. U1 and F1 was found to work well on printouts, however F3 was less useful when printed because the functionality to hold over a ball for more information about the interval is lost.

\subsection{Scenarios}
The scenarios created after focus group 1 were reviewed in the second focus group. After discussing the scenarios with the participants we created a new version:
\vspace{-6mm}
\begin{enumerate}[itemsep=0cm, parsep=0cm]
\item When analysing patients activity level, either individually or in cooperation with occupational therapists or other physiotherapists.
\item In communication with nursing homes and home care personnel.
\item In consultation with the patient.
\item In consultation with next of kin.
\item For educational purposes.
\end{enumerate}

Scenario 1 was changed to include occupational therapists. Occupational therapists are the colleagues that are most often consulted according to the physiotherapists. 

Scenario 2 was changed to include nursing homes as the most important partner to communicate with. Patients in nursing homes are in general much less active than those that live at home and there is therefore a greater need to inform the personnel about how inactive some of their patients are.

Scenario 3 was not changed. It is important to note that the physiotherapists estimated that maybe half of the patients would be cognitively capable of understanding the visualizations.

Scenario 4 was not changed.

Scenario 5 was a new scenario added on the request of the focus group. An increasing part of their work consists of tutoring home care personnel about exercises that patients can perform to increase their activity. The participants felt using visualizations in this setting would be useful. 

\subsection{Requirements}
One requirement which all of the participants wished to clarify was F1-1. They were specific about the fact that it should not be national goals, but national recommendations, this was rectified in the new requirements. Two of the participants mentioned that an overview of the week was good, but receiving a summary of the daily activity was equally important. This was provided by the pie chart but was never explicitly stated as a requirement, F2-9 was added to reflect this.

Being able to include or exclude certain hours of the night was also discussed due it being brought up during several visualizations, F2-10 has been added as a response to this. There was disagreement on what counted as ``night'', and if a certain time interval should be set by the user or a pre-set value should be used. 

\begin{table}[h!]
  \begin{center}
  \begin{tabular}{|c|p{12cm}|}
    \hline
      \textbf{Id} & \textbf{Requirement} \\ \hline
    \multicolumn{2}{|l|}{The visualizations should \ldots} \\ \hline
      F2-1 & give the user an overview of the week where the days are classified by national recommendations or personal goals \\ \hline
      F2-2 & show the activity level for each hour of the day \\ \hline
      F2-3 & make it possible to compare multiple days \\ \hline
      F2-4 & make it easy to identify hours of the day where activity can be added \\ \hline
      F2-5 & show the activity level compared to national or personal goals \\ \hline
      F2-6 & let the user identify patients that are active during the night \\ \hline
      F2-7 & let you compare two separate weeks to see the patients progress \\ \hline
      F2-8 & should be printable in grayscale \\ \hline
      F2-9 & show the activity distribution for a day (sedentary, standing, walking) \\ \hline
      F2-10 & allow the users to toggle if nighttime should be included or not \\ \hline
  \end{tabular}
  \end{center}
  \caption{Functional requirements from the second focus group}
\end{table}

When the participants were presented with the user experience requirements they were initially a bit confused of what exactly these requirements entailed. Some time was used to explain the meaning of UX requirements to the participants. F1-12 should be clarified to include not just the user, but other non medical partners as well, this is now reflected in F2-14. According to the participants some patients are simply not well enough cognitively to understand the visualizations no matter how simple they are. Therefore F1-15 has been changed accordingly.

\begin{table}[h!]
  \begin{center}
  \begin{tabular}{|c|p{12cm}|}
    \hline
      \textbf{Id} & \textbf{Requirement} \\ \hline
    \multicolumn{2}{|l|}{The visualizations should \ldots} \\ \hline
      F2-11 & not be judgemental towards the patients activity level \\ \hline
      F2-12 & should be honest about the patients activity level \\ \hline
      F2-13 & should motivate the patient to be more active \\ \hline
      F2-14 & should be intuitive and easy to understand for the user and third parties \\ \hline
      F2-15 & be easy to explain to cognitively capable patients \\ \hline
      
  \end{tabular}
  \end{center}
  \caption{User experience requirements from the second focus group}
\end{table}

\subsection{Physiotherapy in practice} 
To get an overview of how physiotherapy is conducted in Norway we performed an interview on the participants of the focus group. The physiotherapists that were recruited for the focus groups work with elderly and patients that otherwise needs to be helped from home (e.g. patients suffering from Cerebral Palsy). The seven step process below describes what happens when a patient is referred to a physiotherapist:

\vspace{-4mm}
\begin{enumerate}
  \item A general practitioner or other healthcare personnel can fill out an application for their patients to see a physiotherapist.
  \item The application is then evaluated and placed into a priority queue. Applications may be prioritized if the matter is time sensitive, such as recovery after a fracture or surgery.
  \item When a physiotherapist is available they are given the application on top of the priority queue.
  \item The physiotherapist then makes a house visit to the patient so that they may get an understanding of the current activity level. The activity level is mapped through several different types of exercises and through conversations with the patient. Overall posture and the speed of movements help assess the general state of the patient.
  \item The next step is to create an exercise plan for the patient in order to increase the activity level. During conversations with the patient the physiotherapist discusses what kind of improvements are realistic to achieve considering the current activity level, motivation, physical health etc. All of this data is then used to create an exercise plan that the patient can follow to reach their goals.
  \item When an appropriate plan has been created the physiotherapist returns to the patient to explain how the exercises are executed as well as motivating the patient to reach his goals. The plan may also include other types of activity, for many patients something as simple as walking to the store to buy groceries can be enough to make a difference in the overall activity.
  \item Physiotherapist will return regularly to check up on the patient. The interval between check-ups will vary in respect to how well the physiotherapists expects the patient to follow the agreed upon plan. Some patients lack motivation, and will need more regular check-ups. During such meetings the physiotherapist will get an idea of how much the patient has improved, emphasizing the improvement made is an important factor in motivating the patient to abide to the plan. If there has been little or no change in the activity level, the physiotherapist may want to make changes to the plan so better results are achieved.
\end{enumerate}

\section{Interpretation}
