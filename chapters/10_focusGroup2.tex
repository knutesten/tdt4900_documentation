\chapter{Focus group 2}

\section{Procedure}

\section{Results}

To get an overview of how physiotherapy is conducted in Norway we performed an interview on the participants of the focus group. The physiotherapists that were recruited for the focus groups work with elderly and patients that otherwise needs to be helped from home (e.g. patients suffering from Cerebral Palsy). The seven step process below describes what happens when a patient is referred to a physiotherapist:

\vspace{-4mm}
\begin{enumerate}
  \item A general practitioner or other healthcare personnel can fill out an application for their patients to see a physiotherapist.
  \item The application is then evaluated and placed into a priority queue. Applications may be prioritized if the matter is time sensitive, such as recovery after a fracture or surgery.
  \item When a physiotherapist is available they are given the application on top of the priority queue.
  \item The physiotherapist then makes a house visit to the patient so that they may get an understanding of the current activity level. The activity level is mapped through several different types of exercises and through conversations with the patient. Overall posture and the speed of movements help assess the general state of the patient.
  \item The next step is to create an exercise plan for the patient in order to increase the activity level. During conversations with the patient the physiotherapist discusses what kind of improvements are realistic to achieve considering the current activity level, motivation, physical health etc. All of this data is then used to create an exercise plan that the patient can follow to reach their goals.
  \item When an appropriate plan has been created the physiotherapist returns to the patient to explain how the exercises are executed as well as motivating the patient to reach his goals. The plan may also include other types of activity, for many patients something as simple as walking to the store to buy groceries can be enough to make a difference in the overall activity.
  \item Physiotherapist will return regularly to check up on the patient. The interval between check-ups will vary in respect to how well the physiotherapists expects the patient to follow the agreed upon plan. Some patients lack motivation, and will need more regular check-ups. During such meetings the physiotherapist will get an idea of how much the patient has improved, emphasizing the improvement made is an important factor in motivating the patient to abide to the plan. If there has been little or no change in the activity level, the physiotherapist may want to make changes to the plan so better results are achieved.
\end{enumerate}


\section{Interpretation}
