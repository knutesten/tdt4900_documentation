\chapter{Prototype 2}

% Might have written some things twice. Don't really know how to structure it. Have a look.
\section{Running prototype}
Feedback from the first focus group was used to improve the visualizations, table~\ref{fig:changeLog} shows a list of the changes made to the original system. U2, F2, T2 and T3 were discarded by the first focus group and was removed from the system for the second prototype. 

\begin{table}[h!]
  \centering
  \begin{tabular}{|l|l|p{10.15cm}|}
      \multicolumn{3}{c}{\textbf{Change log}} \\ \hline
      \textbf{Nr} & \textbf{ID} & \textbf{Description} \\ \hline
      1  & U1 & Replaced smiley faces with coloured squares. \\ \hline
      2  & U1 & Classify using national/personal goals for sitting and walking.\\ \hline 
      3  & U2 & Removed. \\ \hline
      4  & F1 & Added pictures to illustrate which activity each slice represents. \\ \hline
      5  & F2 & Removed. \\ \hline 
      6  & F3 & Changed sitting from red to white. \\ \hline
      7  & F3 & Removed nighttime from the dataset. \\ \hline
      9  & T1 & Added goal circles for each timeline. \\ \hline
      10 & T2 & Removed. \\ \hline
      11 & T3 & Removed. \\ \hline
      12 & T4 & Changed sitting colour to white. \\ \hline
      13 & T4 & Reduced inner radius of hour-ticks \\ \hline
      14 & All & Added option to switch between different colours including grayscale. \\ \hline
  \end{tabular}
  \caption{Change log. Changes made to the system after the first focus group.}
  \label{fig:changeLog}
\end{table}

New functionality added includes a new diagram T5, ability to set goals (default is national goals) and a small diagram, goal circles, that shows how active the patient was compared to the goals. The new diagram (see figure~\ref{fig:t5}), T5, is similar to the visualization used by activPAL, but it does not show a bar for the amount of sedentary behaviour. The day is aggregated to 24 hour-blocks, the activity level for each hour is represented by stacked bars. This means that an hour with no activity, other than seating, will show no bar. 

Goal circles is a small diagram appended to T1 and T5, see figure~\ref{fig:t1} and ~\ref{fig:t5}. Goal circles are used to indicate how much activity the patient accumulated compared to the goals set. A full circle means the patient reached his goal that day. Exceeding the goal will produce more circles. For example, if the goal was walking for 1 hour each day, a full circle would indicate that the patient walked exactly 1 hour, a half circle would indicate the patient walked 0.5 hours, and two circles would mean 2 hours of walking etc.

Table~\ref{tab:runProtDesc2} shows an overview of the visualizations used in the second version of the system. A functionality added to all the visualizations was the ability to switch between colours, including grayscale. U2 was discarded by the first focus group. U1, see figure~\ref{fig:uSecond}, is the only week overview for the second version. The smilies were replaced with coloured squares because the sad face would be demotivating for the patients. U1 was also changed to classify the days compared to goals. The goals can be set manually, the default values are the national goals. Reaching both goals will classify the day to the green square, 1 goal to the yellow square, and no goals to the red square.

\begin{table}[h!]
  \centering
  \begin{tabular}{|c|p{11cm}|}
    \hline
    \textbf{ID} & \textbf{Description} \\ \hline
    U1 & Classifies each day of the week into either of three categories: Both goals reached, one goal reached and no goals reached. \\ \hline
    F1 & Pie chart showing the amount of activity for each day. \\ \hline
    F3 & Bubble chart. Divides the pie slices of F1 into bubbles, each bubble representing one interval of activity (e.g. 2 hours of non-stop sedentary behaviour). \\ \hline
    T1 & Timeline of 24 squares, each square represents 1 hour of activity. The amount of activity in that hour is displayed using a gradient. \\ \hline
    T4 & One 24-hour clock showing the activity type using colour coding. \\ \hline
    T5 & Similar to T1, but instead of using a gradient the amount of activity is represented by stacked bars. \\ \hline
  \end{tabular}
  \caption{Visualizations used for focus group 2.}
  \label{tab:runProtDesc2}
\end{table}

// include U figure

F2 was discarded and was removed. The pictures from F2, illustrating the activity type, were added to the new version of F1, see figure~\ref{fig:fSecond}. The feedback on F3 was less concise, so the only change was the colour of walking activity from red to white. This was done because the first focus group expressed concern that all the red would be demotivating.

// include F figure

T2 and T3 was discarded and removed. T1 was not changed other than the goal circle being appended to the right of each timeline, see figure~\ref{fig:t1}. T4, see figure~\ref{fig:t4}, got a colour change from red to white for seated activity, to further highlight actual activity. The inner radius of the hour-ticks, lines that marks each hour, was reduced to make the visualization look more like a normal clock. T5, see figure~\ref{fig:t5}, was the only new visualization added. T5 uses stacked bars to show the activity level of each hour, and uses the goal circle to illustrate the patients activity compared to the goals set.

// include U figure
