\chapter{Analysis of Research methods}
%Tror jeg ``forsvarer'' for mye, burde bare skrive alt vi har gjort feil uten å gjøre no mer kanskje.
Vi burde kanskje omformulere denne seksjonen til "Analysis of research methods" eller noe sånt?


\section{Initial requirements interview}
%Jeg er litt redd for å kalle dette et interview som vi gjør. fordi det var jo egentlig ikke det. Hvis vi først skriver at det var et intervju burde vi i såfall ha skrevet noe om det i HCI kapitellet, som vi kan referere til og hvorfor det skader validiteten. Skal vi heller si ``talks''?
% REVIEW: You have to introduce this section better. You are just saying stuff without introducing what you are talking about. Without context it makes no sense, read sentence 2.
The initial requirements were gathered through informal and casual interviews with a physiotherapist (//we must confirm this//) conducting a research project at St. Olav's Hospital. There was also no structure to the interview. The individual did not work in direct contact with patients, and the data is primarily used in communication with other healthcare personnel. 
%Kanskje ta ut denne biten og heller putte det inn i initial requirements?
They did however have a great deal of experience with visual representations of such data, and communicating through the use of such visual aids.

Due to the background of the interview subject we might not have covered requirements that address concerns when presenting the visualizations in consultation with a patient. 
%Fjerne kanskje denne biten her, der er et ``forsvar'' REVIEW: Should we not defend our research? What is this section anyway?
However we believe that some of these concerns were addressed through additional requirements that arose during the focus group sessions.

%Tror kanskje jeg forsvarer for mye mot slutten, burde kanskje bare konstantere.
% REVIEW: I have to admit I have no idea how this section should look, have you read some of the other master theses to compare?
\section{Focus group}
% REVIEW: We did cover all the user groups, the ones that go home to the patients are the ones that work with old people and they are the ones we wanted to target. 
Nielsen states that the focus group should have at least six participants. We had originally invited 6, but only 5 were able to attend the sessions. We had no issues with keeping the discussion flowing and the participants were all quite forward when they saw something they disliked. We were unable to represent all the relevant user groups as the participants only represented state employed therapists that visited patients at their residence. We have not taken into account physiotherapists that have office hours or work in the private sector.

We deliberately chose to control the navigation through the visualizations ourselves during the focus group, even though Nielsen mentions this as one of two pitfalls that may occur in a focus group. He specifies that by using a demo the user will never have to consider the meaning of screen options or what to do next. In our case the prototype is not an application with intractable buttons, but the prime focus are the visualizations themselves. 

A possible mistake might have been to give a short description of the visualizations to the participants, rather than letting them figure it out on their own, in order to verify if they were intuitive enough. Providing them with a mouse to test the interactivity themselves might have been more beneficial.

\section{Paper Sketches and Prototype}
A mistake we identified once the first focus group was conducted was that we should have worked harder to involve potential users while creating paper sketches. ISO 9241-210 states that one should make an effort to involve users in every step of the development, we failed to do so and the consequences of this have become apparent. If we had presented paper sketches to the users we could have perhaps avoided spending time and resources to create high fidelity prototypes of visualizations that would be quickly discarded.
