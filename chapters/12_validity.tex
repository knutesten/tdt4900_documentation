\chapter{Validity and Reflection on Method}
Addressing and reflecting upon the validity of our research methods is vital in order to show that we are not unaware that no method or execution is without its flaws. This chapter starts off by reflecting over the research methods and finishes off by addressing their validity in accordance to possible threats presented in Section~\ref{subsec:threatsValidity} 

\section{Reflection on Research Method}
An analysis and reflection on how the research methods were executed are contained in this section. The purpose of this chapter is to identify any improvements, mistakes and compare our execution to the theory presented in Chapter~\ref{ch:hci}.

Our overall methodology was based on ISO 9241-210 and its guidelines for performing human-centred design. This ISO is primarily intended as design process when creating commercial interactive systems, but the standard also mentions that it complements a broad variety of development methods. We used an iterative process that was heavily design and user oriented in order to answer our research questions.

In the end we believe this was the correct choice of method, given that we ended up with results that both the participants and researchers were pleased with. After only two focus groups we were able to converge on five visualizations that we and the focus group participants felt could be helpful in aiding physiotherapists complete their tasks more efficiently and with higher quality. Given more time and two or three more iterations with user involvement, we believe that a field ready prototype could be developed and tested.

\subsection{Interviews}
Both of our interviews were unstructured as we had several larger topics we wished to explore. An unstructured interview gave us freedom to pursue relevant topic that might have emerged during the interview, or ask the interviewee to explain their answers further. The downside of not having a clear set of questions is that sub-topics might be overlooked. Once a topic has been discussed and passed, one rarely comes back to it unless some specific questions have been missed. This increases the risk of important information and answers not being pursued enough, or simply overlooked. Unstructured interviews are also hard to compare, generalize and reproduce due to there being no set questions or predefined answers.

The first interview was performed so that we could gain more domain knowledge and assistance in creating a set of initial requirements. The interviewee is a physiotherapist, has a MSc in Human Movement Science and works at St. Olav's hospital conducting a research project. The individual is not in direct contact with patients and the data is primarily used in communication with other healthcare personnel. It might not have been the ideal subject in concern with our intended user group, but we felt that domain expert was satisfactory for a set of initial requirements. She does however have a great deal of experience with representation of data and communicating through visual aids.

The second interview was conducted immediately after the final focus group. The purpose of this interview was so that the authors could learn how the participants conducted physiotherapy. One of the participants was interviewed, but due to it being immediately after the focus group two participants were still presented and wished to provide input. Once we had finished with the original interviewee the other two elaborated on things they felt we missed or wished to explain in more detail. Because there were other participants present during the interview it might have made the interviewee biased because he did not want to look bad around his peers, or it might have altered his behaviour. Having them in one at a time may have perhaps been ideal, but it would in turn have broken the open environment created by the focus group and felt more artificial.

\subsection{Focus group}
Nielsen (see Section~\ref{sec:focusGroup}) states that a focus group should have at least six participants. We had originally invited six, but only five were able to attend the sessions. Even with five participants there were no issues with keeping the discussion flowing and the participants were all quite forward when they saw something they disliked. We were unable to represent all the relevant user groups as the participants only represented state employed therapists that visited patients at their residence. We did not include possible patients, or privately employed therapists who have elderly patients.
% REVIEW: Are you certain that the above is a bad thing, because the ones that work with older people are the ones that work for Trondheim Kommune. Unless they are elderly, but don't have health problems.

We deliberately chose to control the navigation through the visualizations ourselves during the focus group, even though Nielsen mentions this as one of two pitfalls that may occur in a focus group. He specifies that by using a demo the user will never have to consider the meaning of screen options or what to do next. In our case the prototype is not an application with intractable buttons, but the prime focus are the visualizations themselves. At least one more focus group should have been conducted as new requirements did surface which should ideally have been prototyped and evaluated due to time constraints this was not possible.

A possible mistake might have been to give a short description of the visualizations to the participants, rather than letting them figure it out on their own, in order to verify if they were intuitive enough. Providing them with a mouse to test the interactivity themselves might have been more beneficial. One of the participants talked more then the others, and might have partly influenced the other participants to some degree. There were situations where other participants disagreed with the dominant one and was even swayed in some cases, so complete influence over other members was not the case.

\subsection{Brainstorming and Prototype}
A guideline given both by ISO 9241-210 and Section~\ref{sec:brainstorming} that we failed to follow was to include participants from a wide range of disciplines and potential users in the Brainstorming. This mistake became apparent after the first focus group was conducted where we were forced to completely discard certain visualizations because the presentation style was not useful to the physiotherapists. Had this been identified earlier we would not have had to spend time and resources implementing high fidelity prototypes that would be quickly discarded.

According to Houde and Hill a prototype should be placed in the triangle (see Section~\ref{sec:prototypesPrototype}) based on what questions s and design decisions it seeks to answer. Our running prototypes have been primarily focused on the look and feel, but have sufficed in answering any implementation uncertainties we might have had. This means that design questions relating to the role of our product remain unproven. We know what the role of our final system is and what the physiotherapists expect of it based on feedback received in the focus groups, but the prototype was never tested as part of the potential users routine and workflow. This means that there might still be undiscovered potential as to how it can be used, and possible design issues remain undiscovered.

\section{Validity of Research}
This section discusses the validity of our research methods based on the threats presented in Section~\ref{subsec:threatsValidity}.

\subsection{Construct Validity}
%In the overall plan for our research (section~\ref{sec:overview}) we present an outline on how we planned to conduct our research. Table~\ref{tab:designPlan} shows correlation between research method and research questions. 

Focus groups are a common technique to provide feedback on ideas or prototypes, as well as provide new ideas and suggestions for improvement. Given the fact that there are few similar systems to compare with, and the small amount of information about the practice of physiotherapy in Norway, focus groups are in our opinion a good method to use given the time constraints and circumstances.

One common factor that causes threats to construct validity in focus groups and interviews are participants or interviewees misinterpreting questions being asked by the moderator or interviewer. To reduce this threat we tried to make sure that the participants understood the questions, and encouraged them to repeat them or provide a response to confirm that this was the case.

\subsection{Internal Validity}
Participants answering questions in a way that make them look better (social desirability bias), leading questions by the moderators and cues unconsciously given to the participants all contribute to hurting the internal validity of the research methods. We have tried to avoid leading question by wording ourselves carefully during the interviews and focus groups, and encouraged participants to be as honest as possible, but can not exclude that unconscious influence by us or other participants has taken place.

\subsection{External Validity}
Factors such as bias, group thinking, and an artificial setting make focus group results hard to generalize and suffer from low external validity. The two main factors that hurts our external validity, other than the choice of research method, are the unstructured nature of our interviews and the small amount of participants in our focus groups. The focus groups were the main source of our results, and having more participants would improve their validity.

As hard as it is to generalize unstructured qualitative methods, we believe that our findings hold some merit. Considering the positive feedback we got from the participants of the focus groups, we think that the results presented in this report can be generalized for physiotherapists who perform home or clinical visits. We are however unsure if our results can be applied to physiotherapists that have office hours, as they were not included in any of the focus groups or interviews.

\subsection{Reliability}
The issue with focus groups is that they are not very reliable, because it is based on subjective opinions and interpretations made by the researchers and participants. We have done our best to make the research methods as reliable as possible. Our interview process was highly unstructured, we are aware that this hurts the external validity and makes it hard to reproduce, but we hope that the description in Section~\ref{sec:reqGathering} will make up for this. 

How the research methods are executed is well described throughout the report, and our prototype has also been handed in as a part of the deliverables as an attempt to increase the overall reliability. The reliability of our prototypes should be quite high, as the code is provided with this thesis, and both versions of the running prototype are included.
