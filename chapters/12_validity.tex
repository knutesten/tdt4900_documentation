\chapter{Reflection of Research Methods}
An analysis and reflection on how the research methods were executed are contained in this chapter. The purpose of this chapter is to identify any improvements, mistakes and compare our execution to the theory presented in chapter~\ref{ch:hci}.
%Tror jeg ``forsvarer'' for mye, burde bare skrive alt vi har gjort feil uten å gjøre no mer kanskje.

\section{Interviews}
Both of our interviews were unstructured as we had a particular topic we wished to explore. This gave us a large amount of freedom to receive elaborating on particular replies or steer the direction of the interview to interesting topics that came to light. The downside of not having a clear set of questions but only a large topic is that sub-topics might be overlooked or missed. Once a topic has been discussed and has passed, one rarely comes back to it unless some specific questions have been missed. This increases the risk of important information or answered not being pursued enough, or simply overlooked. Unstructured interviews are also hard to compare, generalize and reproduce due to there being no set questions or pre defined answers.

The first interview was performed so that we could receive more domain knowledge and assistance in creating a set of initial requirements. The interviewee is a physiotherapist, has a MSc in Human Movement Science and works at St. Olav's hospital conducting a research project. The individual is not in direct contact with patients and the data is primarily used in communication with other healthcare personnel. It might not have been the ideal subject in concern with our intended user group, but we felt that they were satisfactory for a set of initial requirements. They do however have a great deal of experience with representation of data and communicating through visual aids.

The second interview was conducted immediately after the final focus group. The purpose of this interview was so that the authors could learn how the participants conducted physiotherapy. One of the participants was interviewed, but due to it being immediately after the focus group two participants were still presented and wished to provide input. Once we had finished with the original interviewee the other two elaborated on things they felt we missed or wished to explain in more detail. Because there were other participants present during the interview it might have made the interviewee biased because he did not want to look bad around his peers, or it might have altered his behaviour. Having them in one at a time may have perhaps been ideal, but it would in turn have broken the open environment created by the focus group and felt more artificial.

\section{Focus group}
Nielsen (see section~\ref{sec:focusGroup}) states that a focus group should have at least six participants. We had originally invited 6, but only 5 were able to attend the sessions. Even with 5 participants there were no issues with keeping the discussion flowing and the participants were all quite forward when they saw something they disliked. We were unable to represent all the relevant user groups as the participants only represented state employed therapists that visited patients at their residence. We did not include possible patients, or privately employed therapists who have elderly patients.
% REVIEW: Are you certain that the above is a bad thing, because the ones that work with older people are the ones that work for Trondheim Kommune. Unless they are elderly, but don't have health problems.

We deliberately chose to control the navigation through the visualizations ourselves during the focus group, even though Nielsen mentions this as one of two pitfalls that may occur in a focus group. He specifies that by using a demo the user will never have to consider the meaning of screen options or what to do next. In our case the prototype is not an application with intractable buttons, but the prime focus are the visualizations themselves. At least one more focus group should have been conducted as new requirements did surface which should ideally have been prototyped and evaluated, due to time constraints this was not possible.

A possible mistake might have been to give a short description of the visualizations to the participants, rather than letting them figure it out on their own, in order to verify if they were intuitive enough. Providing them with a mouse to test the interactivity themselves might have been more beneficial. One of the participants talked more then the others, and might have partly influenced the other participants to some degree. There were situations where other participants disagreed with the dominant one and was even swayed in some cases, so complete influence over other members was not the case.

\section{Brainstorming and Prototype}
A guideline given both by ISO 9241-210 and section~\ref{sec:brainstorming} that we failed to follow was to include participants from a wide range of disciplines and potential users. This mistake became apparent after the first focus group was conducted where we were forced to completely discard certain visualizations because the fundamental idea was not liked by the physiotherapists.

We identified this mistake once the first focus group was conducted, we should have worked harder to involve potential users or physiotherapists in the creation of paper sketches. ISO 9241-210 states that one should make an effort to involve the users in every step of the development, we failed to do so and the consequences became apparent.

A mistake we identified once the first focus group was conducted was that we should have worked harder to involve potential users while creating paper sketches and brainstorming. ISO 9241-210 states that one should make an effort to involve users in every step of the development, we failed to do so and the consequences of this are apparent. If we had presented paper sketches to the users we could have perhaps avoided spending time and resources to create high fidelity prototypes of visualizations that would be quickly discarded.
