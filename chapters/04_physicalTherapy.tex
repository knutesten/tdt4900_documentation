\chapter{Physical therapy for the elderly}

\section{Physical activity and health}
In 2010 one of the largest ever systematic efforts to describe the global health situation was conducted. The article was later published in The Lancet \cite{globalBurden}. One of their many findings was that since 1970 men and women have gained an additional ten years to their life expectancy, but spend more time living with injuries and illness. 

The amount of time adults spend in a sedentary position has increased over the last 30 years. The reasons for this are many, but increased use of technology and ease of transport are one of the main factors \cite{sedentaryBehaviour}. An American study shows that 1 of 4 US adults spend 70\% of their waking hours in a sedentary position, 30\% in light activity, and little to no time is spent exercising. During the last decade research has started to emerge that links extended periods of sedentary time to metabolic risks\cite{sedentaryTime}, obesity, and abnormal glucose metabolism \cite{breaksSedentary}. It is suggested that prolonged periods of sedentary time should be avoided by increasing the number of breaks during sedentary time, and these findings suggest new public health recommendations \cite{breaksSedentary}. Another study goes as far as stating that prolonged sedentary time is strongly related to metabolic risks independent of physical activity \cite{sedentaryActivity}, and that older people benefit more from reducing the sedentary time. What has not been shown by research is how long a subject can stay sedentary before it has negative impact on the individuals health, or how long the breaks between sedentary time should be.

The \gls{ndh} has issued recommendations and guidelines pertaining the minimum amount of physical activity an elderly person should have. The recommendation is set to a minimum of 30 minutes a day with moderate activity, a brisk walk is considered a moderate activity ~\cite{helsedirektoratetFysiskAktivitet}.

//More about why it is important to move around. could mention farseeing but we ar not part of it. WHere should we put in the national limits and all that?

FROM THE OLD REPORT. REMEMBER TO USE THINGS FROM THE OLD REPORT

\section{Physical therapy in Norway}
//Scenarios what do they do. How do they work.

\section{Sensor technology in physical therapy}
//Find articles about technology used by this group

\section{Potential of AMU}
