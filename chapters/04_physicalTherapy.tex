\chapter{Physical therapy for the elderly}

\section{Physical activity and health}
In 2010 one of the largest ever systematic efforts to describe the global health situation was conducted. The article was later published in The Lancet~\cite{globalBurden}. One of their many findings was that since 1970 men and women have gained an additional ten years to their life expectancy, but spend more time living with injuries and illness. 

The amount of time adults spend in a sedentary position has increased over the last 30 years. The reasons for this are many, but increased use of technology and ease of transport are one of the main factors~\cite{sedentaryBehaviour}. An American study shows that 1 of 4 US adults spend 70\% of their waking hours in a sedentary position, 30\% in light activity, and little to no time is spent exercising. During the last decade research has started to emerge that links extended periods of sedentary time to metabolic risks~\cite{sedentaryTime}, obesity, and abnormal glucose metabolism~\cite{breaksSedentary}. It is suggested that prolonged periods of sedentary time should be avoided by increasing the number of breaks during sedentary time. 

Based on these findings health recommendations regarding breaks in sedentary time should be added to the already existing ones~\cite{breaksSedentary}. Another study goes as far as stating that prolonged sedentary time is strongly related to metabolic risks independent of physical activity~\cite{sedentaryActivity}, and that elderly benefit more from reducing time spent sedentary than younger people. Though researchers agree that long periods of sedentary time is unhealthy, research still has to be done on how long a subject can stay sedentary before it has negative impact on the individuals health. How long you need to stay active between periods of sedentary time is also debated.

The \gls{ndh} has issued recommendations and guidelines pertaining to the minimum amount of physical activity for en elderly person. The recommendation is set to a minimum of 30 minutes a day with moderate activity~\cite{helsedirektoratetFysiskAktivitet}. In addition to this an elderly person should be standing in a skeleton bearing position for total of 5 hours per day to preserve the skeletons strength and form. Skeleton bearing position means that the subject is standing upright without any aid, such as a walking stick. (WE NEED A SOURCE!!! ASK PHYSIOTHERAPISTS)

\section{Physical therapy in Norway}
To get an overview of how physiotherapy is practiced in Norway we conducted an interview with the participants of the focus group. To clarify the process described by the participants we have devided it into 7 steps:
\vspace{-4mm}
\begin{enumerate}
  \item A general practitioner or other healthcare personnel can create an application for their patients to see a physiotherapist.
  \item The application is then evaluated placed into a priority queue. Some applications will get prioritized if there is especial need for physiotherapy, for example after a fracture or operation.
  \item When a physiotherapist is free, he is given the application on top of the priority queue. 
  \item The physiotherapist then goes home to the patient to get an idea of the current activity level. The activity level is mapped through several different types of exercises and through conversations with the patient. Posture and the speed of movement can also be indicators on the general state of the patient.
  \item The next step is to create an exercise plan for the patient to increase the activity level. During the conversations with the patient the physiotherapist discusses what kinds of improvement is realistic to achieve considering the current activity level, motivation of the patient etc. All this data is then used to create an exercise program that the patient can follow to reach his goals.
  \item When an appropriate plan has been created the physiotherapist returns to the patient to explain how the exercises are executed as well as motivating the patient to reach his goals. The plan may also include other types of activity other than exercises, for many patients something as simple as walking to the store to buy groceries can be enough to make a difference in the overall activity.
  \item After a time the physiotherapist will return regularly to check up on the patient. The interval between checkups will vary in respect to how well the physiotherapists expects the patient to follow the agreed upon plan. Some patients lack motivation, and will need more regular checkups. During this meeting the physiotherapist will get an idea of how much the patient has improved, emphasizing the improvement made is an important factor in motivating the patient to continuing to stick to the plan. If there has been little or no change in the activity level, the physiotherapist may want to make changes to the plan to get better results.
\end{enumerate}

\section{Sensor technology in physical therapy}
//Find articles about technology used by this group

\section{Potential of AMU}
