\chapter{Physical therapy for the elderly}

\section{Physical activity and health}
In 2010 one of the largest ever systematic efforts to describe the global health situation was conducted. The article was later published in The Lancet~\cite{globalBurden}. One of their many findings was that since 1970 men and women have gained an additional ten years to their life expectancy, but spend more time living with injuries and illness. 

The amount of time adults spend in a sedentary position has increased over the last 30 years. The reasons for this are many, but increased use of technology and ease of transport are one of the main factors~\cite{sedentaryBehaviour}. An American study shows that 1 of 4 US adults spend 70\% of their waking hours in a sedentary position, 30\% in light activity, and little to no time is spent exercising. During the last decade research has started to emerge that links extended periods of sedentary time to metabolic risks~\cite{sedentaryTime}, obesity, and abnormal glucose metabolism~\cite{breaksSedentary}. It is suggested that prolonged periods of sedentary time should be avoided by increasing the number of breaks during sedentary time. 

% REVIEW: It sounds strange to me to say that "these findings suggest". Dean: FIXED
Based on these findings health recommendations regarding breaks in sedentary time should be added to the already existing ones ~\cite{breaksSedentary}. Another study goes as far as stating that prolonged sedentary time is strongly related to metabolic risks independent of physical activity~\cite{sedentaryActivity}, and that elderly benefit more from reducing time spent sedentary than younger people. Though researchers agree that long periods of sedentary time is unhealthy, research still has to be done on how long a subject can stay sedentary before it has negative impact on the individuals health. How long you need to stay active between periods of sedentary time is also debated.
%REVIEW: ^ review last part which I have changed. Dean: OK

The \gls{ndh} has issued recommendations and guidelines pertaining to the minimum amount of physical activity for en elderly person. The recommendation is set to a minimum of 30 minutes a day with moderate activity~\cite{helsedirektoratetFysiskAktivitet}. In addition to this an elderly person should be standing in a skeleton bearing position for total of 5 hours per day to preserve the skeletons strength and form. Skeleton bearing position means that the subject is standing upright without any aid, such as a walking stick. (WE NEED A SOURCE!!! ASK PHYSIOTHERAPISTS)
%REVIEW: Read the last section, it has been changed. Hopefully for the better. Dean: OK


\section{Physical therapy in Norway}
//Scenarios what do they do. How do they work.

\section{Sensor technology in physical therapy}
//Find articles about technology used by this group

\section{Potential of AMU}
\part{title}
