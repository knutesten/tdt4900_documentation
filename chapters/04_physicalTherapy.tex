\chapter{Physical therapy for the elderly}

\section{Physical activity and health}
In 2010 one of the largest ever systematic efforts to describe the global health situation was conducted. The article was later published in The Lancet~\cite{globalBurden}. One of their many findings was that since 1970 men and women have gained an additional ten years to their life expectancy, but spend more time living with injuries and illness. 

The amount of time adults spend in a sedentary position has increased over the last 30 years. The reasons for this are many, but increased use of technology and ease of transport are one of the main factors~\cite{sedentaryBehaviour}. An American study shows that 1 of 4 US adults spend 70\% of their waking hours in a sedentary position, 30\% in light activity, and little to no time is spent exercising. During the last decade research has started to emerge that links extended periods of sedentary time to metabolic risks~\cite{sedentaryTime}, obesity, and abnormal glucose metabolism~\cite{breaksSedentary}. It is suggested that prolonged periods of sedentary time should be avoided by increasing the number of breaks during sedentary time. 

Based on these findings health recommendations regarding breaks in sedentary time should be added to the already existing ones~\cite{breaksSedentary}. Another study goes as far as stating that prolonged sedentary time is strongly related to metabolic risks independent of physical activity~\cite{sedentaryActivity}, and that elderly benefit more from reducing time spent sedentary than younger people. Though researchers agree that long periods of sedentary time is unhealthy, research still has to be done on how long a subject can stay sedentary before it has negative impact on the individuals health. How long you need to stay active between periods of sedentary time is also debated.

<<<<<<< HEAD
The \gls{ndh} has issued recommendations and guidelines pertaining to the minimum amount of physical activity for en elderly person. The recommendation is set to a minimum of 30 minutes a day with moderate activity~\cite{helsedirektoratetFysiskAktivitet}. In addition to this an elderly person should be standing in a skeleton bearing position for total of 5 hours per day to preserve the skeletons strength and form. Skeleton bearing position means that the subject is standing upright without any aid, such as a walking stick.
=======
The \gls{ndh} has issued recommendations and guidelines pertaining to the minimum amount of physical activity for en elderly person. The recommendation is set to a minimum of 30 minutes a day with moderate activity~\cite{helsedirektoratetFysiskAktivitet}. In addition to this an elderly person should be standing in a skeleton bearing position for total of 5 hours per day to preserve the skeletons strength and form. Skeleton bearing position means that the subject is standing upright without any aid, such as a walking stick.  
>>>>>>> 8e19c8c261d172e451f7ef10b8e8a7215cf07cd1

\section{Physical therapy in Norway}

\section{Sensor technology in physical therapy}
According to the participants of the focus groups as well as other physiotherapists we have been in contact with, there is no use of sensor technology in physiotherapy in Norway today. One of the participants worked on a research project where the activPAL sensor was used to track the activity of hip-fracture patients, but had never used it outside of academic work.

% OLD SECTION. SAVED IT IN CASE WE NEED TO WRITE SOMETHING LIKE THIS LATER.
%Though only one of the participants had actual experience using sensor data related to work, all the participants were very positive to the idea of using this type of technology in their own practice. A questionnaire answered during the focus groups revealed that all the participants wanted to make use of more technology in their work. All the participants also believed that the effectiveness and quality of their work would increase if such technology were to be used. A more thorough survey is needed to conclude that this is an attitude shared by the majority of physiotherapists.

%Norwegian articles stating the interest in use of technology in recovery would be nice. Not saying these are good example but is stuff out there: http://www.forskning.no/artikler/2011/februar/278704 http://www.forskning.no/artikler/2012/februar/313131

\section{Potential of IMU}

% OLD SECTION. SAVED IT IN CASE WE NEED TO WRITE SOMETHING LIKE THIS LATER.
%Looking at the seven step scenario of the previous section, sensor technology such as activPAL can be used in step 4 to help gather data about the current activity level of the patient. Such data will in many cases be more accurate, because it can show the activity over an entire week or weeks.

%Visualizations such as the ones created in this project can be used in steps 5-7. In step 5 the physiotherapist can use the detailed data to identify important factors such as which hours of the day are the least active, and if the patient is active during the night. Such data can otherwise only be gathered through talking with the patient, which is often inaccurate. This type of information can be very helpful when creating the exercise plan.

%In step 6 the visualizations can be used to explain the current activity level. Looking at different diagrams can help the patients understand how and why then need to be more active. Some of the physiotherapists stated that having quantitative statistical diagrams would be helpful in convincing patients that there was need for improvement. It was also stated that having statistical diagrams would give them more credibility, because people tend to trust statistics. 

%Visualizations can also be a great help in step 7, as a motivational tool. Often patients do not notice the improvement at first, in such cases diagrams can show the improvement that the patient does not notice by himself. Feeling a sense of progress is one of the most important factors for patients to be motivated to stick to the exercise plan.
