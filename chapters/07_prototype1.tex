\chapter{Prototype 1}

\section{activPal}

\section{Paper sketches}

\subsection{Fractional charts}
These type of charts show the sum of time spent sedentary, standing, and walking. Summing over the three classifications makes it easy to get an overview of the day as a whole. Though the summation gives a great overview, details are lost and it is not possible to pinpoint when each activity occurred during the day. Fractional charts can be useful for quickly identifying days with a low amount of activity, before spending time looking at more detailed visualisations. It might also serve as an alert for subjects with a low activity level.

\subsubsection{Pie chart}
The pie chart is a standard way of showing fraction of time spent on each behavioural pattern. The overall design of the pie chart is very standard. We decided to create a separate legend instead of writing a descriptive text on different pie slices. The legend is a simple box which connects a behavioural pattern to each colour. The exact percentage for each behavioural pattern is also displayed.

\subsubsection{Symbolic}
A more symbolic approach is to remove the legend and instead use illustrations so that the user intuitively understands what each colour or box corresponds to. Since the illustrations are had to fit into a pie slice this chart uses scaled boxes instead of a circle to represent the fraction. 

%Dette burde kanskje flyttes til running prototype? Sammen med andre ting vi fant ut ikke funket like bra i praksis som papir.
The first idea was to only use the illustrations \ref{fig:symbolicPie}, without the boxes, and scale the height to represent the ratio between each activity classification. However this idea was discarded because it became apparent that it was too hard to identify the ratio between the different illustrations. This is much easier with boxes.

%We may have to redraw this picture compared to the explanation in this section?
\begin{figure}[h!]
	\centering
		\includegraphics[width=0.7\textwidth]{stickSketch.png}
		\caption{\footnotesize Rough drafts of a symbolic "pie chart"}
		\label{fig:symbolicPie}
\end{figure}

\subsubsection{Ball chart}
Another approach, is to include some more details into a pie chart style, by using balls instead of normal pie slices. Each ball represents an interval of one of the classifications, so that many balls of one colour both represents the amount of that behaviour and shows how long each period of that behaviour was. Figure %need a figure here
shows an example of such a graph. The benefit with this type of graph is that you easily can identify if you have long stretches of sedentary behaviour. Taking small breaks with active behaviour can help ``split up'' those balls, which may be beneficial for your health. Adding interactivity to the chart you can select each ball and see what time it corresponds to. This way you can quickly find the larger balls and identify what time of the day this behaviour occurred. 
%I don't even know man, fuck you.

\subsection{Timeline}
Timeline visualisations are effective at illustrating when various activities occurred during the day. The illustration uses a long horizontal bar that has different colours for different behavioural classification. Such a bar can be used to identify points during the day where the subject is in a sedentary position for too long. By looking at multiple days, the user can detect patterns in the day where he needs to be more active.

\subsubsection{Continuous}
One approach to this visualization is to create a continuous timeline that contains every little detail of behaviour, see figure \ref{fig:timelineContinuous}. The continuous timeline is useful for quickly identifying periods of the day with unsatisfactory behaviour, but the detail can also be distracting and make it harder to read the graph. Adding a fourth colour to highlight long periods of sedentary behaviour can make the timeline easier to interpret.

%Perhaps represent the block based one instead?
\begin{figure}[h!]
	\centering
		\includegraphics[width=0.7\textwidth]{continousTimelineSketch.png}
		\caption{\footnotesize Timeline with hour blocks}
		\label{fig:timelineContinuous}
\end{figure}
%Does this part need more of an introduction?
%Why would you make it less formal?
%Skrevet det litt om for å gjøre det mindre formelt
We came up with a suggestions to represent the passing of time during a day by using an animated timeline and stick figures. The timeline will be drawn in real time while stick figures simultaneously perform the corresponding activities. By displaying the day gradually we hope that the subject will gain a firm understanding of their day. This means that this visualization can not be used to gain a quick overview, but is intended to be used when viewing a day for the first time.

The more motivational approach would be to replace the stick figure with an analogy or metaphor. Instead of a stick figure, a flower could be used. Activity would allow the plant to get sunlight, making it grow. Sedentary positions would make the weather cloudy and the flower would be unaffected.

\subsubsection{Blocks}
Instead of having a continuous scale, a blocked approach can be used, as seen in figure \ref{fig:timelineBlocks}. The timeline would be divided into 24 blocks, each block corresponding to an hour. A gradient colour scale would be used to represent the amount of activity inside the hour block. This should make it easy to identify hours in the day where prolonged sedentary positions are present. Giving feedback about specific hours might make it easier to interpret and make use of the chart, because you are alerted to certain hours of the day where you should be more active.

%Perhaps represent the block based one instead?
\begin{figure}[h!]
	\centering
		\includegraphics[width=0.7\textwidth]{timelineBlocksSketch.png}
		\caption{\footnotesize Timeline with hour blocks}
		\label{fig:timelineBlocks}
\end{figure}

\subsubsection{Clock}
%Is this part too short now?
A timeline may need some explanation before the user understands it properly. By creating two clocks instead of a long horizontal bar the user can more intuitively understand what the visualization is presenting. Since a clock has only 12 hours, two clocks would need to be drawn. To make it easier to identify day and night, a descriptive background will be added, see figure \ref{fig:clock12}.

\begin{figure}[h!]
	\centering
		\includegraphics[width=0.7\textwidth]{clock12Sketch.png}
		\caption{\footnotesize Two 12 hour clocks show the activity of the day.}
		\label{fig:clock12}
\end{figure}

Another approach is to use one 24 hour clock, see figure \ref{fig:clock24}. This makes it easier to see the transition between AM and PM, but the users will not be used to seeing a 24 hour clock. This clock will also utilize different background to differentiate between day and night more easily.

\begin{figure}[h!]
	\centering
		\includegraphics[width=0.5\textwidth]{placeholder.jpg}
		\caption{\footnotesize A single 24 hour clocks show the activity of the day.}
		\label{fig:clock24}
\end{figure}

%Might add a paragraph about highlighting long periods of sedentary behaviour.

\subsection{Week overview}
Getting an overview of the week as a whole can be useful as an introduction. By looking at an overview the user can quickly identify problem days that can then be investigated further. These charts could also be used as the top level of an interactive application. Each day could then be clicked to show either a timeline or pie chart. 

\subsubsection{Day classification}
By calculating the overall activity level and classifying the days into three categories the user can easily see which days he need to be more active and which days the activity level is satisfactory. In our sketch, see figure \ref{fig:smileyWeek} the three different classifications are illustrated by smilies (smiling face for active days, and sad face for inactive days).

\begin{figure}[h!]
	\centering
		\includegraphics[width=0.7\textwidth]{smileyWeekSketch.png}
		\caption{\footnotesize Week overview with each day classified into one of three categories.}
		\label{fig:smileyWeek}
\end{figure}

A more complex version of the above chart, see figure \ref{fig:detailedWeek} is to show a square for each day, while still using the same classification into sad and happy smilies. Each day square will then contain 24 smaller squares that represent each hour of the day. The small hour square are coloured with a gradient to show the activity level that hour.

\begin{figure}[h!]
	\centering
		\includegraphics[width=0.6\textwidth]{detailedWeekSketch.png}
		\caption{\footnotesize Some explanation here.}
		\label{fig:detailedWeek}
\end{figure}

With this chart you can get an overview of the week as a whole, and identify what hours of the inactive days had the most sedentary behaviour. 

%This should be moved I guess, or the section should be renamed away from prototype to something like "Graphs" I am not sure.
\section{Programming Framework}
%Do we need to add some text before the HTML5 subsection?

\subsection{HTML5}
HTML is a markup language for the creation of web pages. HTML describes the structure and the contents of the web page. In later years, the need for advanced styling and complex interaction with web pages has made CSS and JavaScript increasingly popular. HTML5 was created as a response to this, HTML5 is an umbrella term for creating web pages using HTML5, CSS3 and JavaScript.

HTML5 has simplified the syntax compared to earlier versions. New tags have been added to better represent the modern web page elements. Other features include media tags which greatly simplifies adding multimedia content, such as playing audio and video files. More importantly for our project is the extensive support for interactive and animated graphics through the \emph{canvas-} and \emph{svg}-tag.

The new features of HTML5 and CSS3 make it much easier to create web applications for multiple platforms and screen sizes. After the smartphone and tablet revolution, creating responsive and adaptable websites has become more important. The new features included in HTML5 give large amount of flexibility with respect to the user interface and graphical visualizations.

% This may not be very relevant.
\emph{Cascading Style Sheets} (CSS) is a language used to describe the styling of an HTML document. CSS documents describes the size, color and look of HTML elements. A new feature in CSS3, which is part of HTML5, is \emph{Media Queries}. With Media Queries it is possible to specify different styling relative to the size of the screen. This functionality is useful when creating applications that target devices with different screen sizes, such as smartphones, tablets and laptops. 

\subsection{Data-Driven Documents}
JavaScript is the main scripting language for web pages. It is a client-side scripting language that allows programmers to add functionality to otherwise static HTML-pages. While CSS3 takes care of the styling of HTML-elements, JavaScript is used to create customized behaviour. All modern browsers have JavaScript engines/interpreters that compile and run JavaScript code.

JavaScript is now an industry standard maintained by ECMA International. The standardized version of the script is named ECMAScript. Today, the names ECMAScript and JavaScript are used interchangeably, and JavaScript is often used to refer to ECMAScript. Because different browsers have different implementations of the JavaScript engine, slight variations in the way JavaScript code will run on these browsers exists.

Together with HTML5 and CSS3, JavaScript is great for creating web applications that can be designed to run on both mobile and stationary devices. JavaScript has a multitude of useful open source libraries that can be used to create complex user interaction, animation, and custom graphics.

One of the challenges in this project was to create different visualizations to represent the activity patterns of subject. Creating custom graphics in HTML5 can be done using both the canvas- and the svg-tag. In this project \gls{svg} is used. \gls{svg} is an image format that uses XML encoding to define shapes, lines, colors, and text. One benefit of this, compared to other image formats, is that details in \gls{svg}-images will not be lost when zooming. All popular browsers, and most mobile devices, support rendering of the \gls{svg}-images.

Creating graphics using svg-tags directly is cumbersome and time consuming. \gls{d3} is an open source framework that greatly simplifies this task. \gsl{d3} is written in JavaScript and designed to be used in combination with HTML5. The framework can be used both to create new \gls{svg} images from scratch or modify and edit existing images. Another feature is the ability to easily add interactivity and animation to the \gls{svg}-elements. HTML5 in combination with \gls{d3} will gives us a flexibility to create almost any type of visualization and adding interactivity and animation to it.

\section{Running prototype}
