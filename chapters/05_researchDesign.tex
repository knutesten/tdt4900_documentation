\chapter{Research Design}


\section{Overall plan}
%I would like to include the name or section where this method is used, but I am not sure where to exactly put it. I guess I can put it into the sections that justify and talk about the method.
This section provides and overview of how our chosen research methods relate to each other and how they to the research questions we have posed. Table ~\ref{tab:designPlan} gives a quick summary of our methods, their purpose and how they relate to our research questions. The order in the table reflects in what order they appear in the report and were conducted.

\begin{table}[h!]
  \centering
  \begin{tabular}{|p{0.7cm}|p{2cm}|p{9cm}|}
    \hline
    \textbf{RQ} & \textbf{Method} & \textbf{Description} \\ \hline
    1,2 & Interview & An interview was performed with a domain expert which would allows us to establish initial requirements. \\ \hline
    2 & Brainstorming & Based on the initial requirements the authors performed brainstorming sessions among themselves and sketches were created. \\ \hline
    2 & Prototype & The paper sketches were used as a starting point to create a high fidelity prototype that would be presented to first focus group. \\ \hline
    1,2,3 & Focus group & The first running prototype was presented to the focus group and would enable us to refine it further. \\ \hline
    1,2 & Prototype & Based on the feedback received from the first focus group the prototype was iterated. \\ \hline
    None & Questionnaire & Before the final focus group a quick questionnaire was answered by the users. \\ \hline
    1,2,3 & Focus Group & The new iteration of the prototype was presented and the requirements were refined. \\ \hline
  \end{tabular}
  \caption{The overall design plan}
  \label{tab:designPlan}
\end{table}

 More detailed information about how the methods were executed and their result can be found in the subsequent chapters of this report, namely in Chapter 6 to 10. The following sections of this Chapter explain the rationale and choice of method in further detail.
 
\section{Interview}
 Initially we had no domain knowledge or experience with how visualizations could be used to present sensor data. So when the time came to outline initial requirements and scenarios we realised that we needed help. While attempting to acquire a set of activPal sensor from St. Olav's Hospital we came in contact with a (what is she!?) who had extensive knowledge about activPal and visualization of sensor data. We considered performing a field study, a focus group or an interview. In the end we chose to interview the domain expert. We believed that a field study and focus group would not be fruitful enough to justify the amount of time it would take. We lacked a grasp of basic knowledge and terminology in the field that could be covered by a simple interview, without being confused or influenced by the multitude of opinions that would appear in a focus group or field study. The procedure and results of this interview can be found in Chapter~\ref{ch:initialRequirements} 
 
\section{Brainstorming}
 