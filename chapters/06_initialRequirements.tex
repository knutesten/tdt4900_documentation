\chapter{Initial requirements}

\section{Project context}
//Farseeing fall and elderly


% I don't even know what to think about this section. Please review and add stuff if you get some ideas. I didn't really participate much in this process. Section should probably be a little longer... HELP ME DEAN HEEEELP ME!
\section{Requirement gathering}
The initial requirements were gathered with the help of a physiotherapist working on a research project at St. Olavs Hospital. Here the activPAL sensor is used to track the activity of patients being rehabilitated after hip-fractures. The only visualization utilized is the one offered by the activPAL software.

Research projects such as the one mentioned above is currently the only arena where physiotherapists use \gls{pi} technology to track the activity level of their patients. We therefore expected the requirements to be changed considerably after the two focus group sessions. 

The first version of the requirements were created during the first meeting. Questions and issues that arose during the development phase were solved by discussion via mail or telephone.

\section{Initial Requirements and User Scenario}
\begin{table}[h!]
  \begin{center}
  \begin{tabular}{|l|p{12cm}|}
    \hline
      \textbf{Id} & \textbf{Requirement} \\ \hline
      1 & The visualizations should give an overview of the week \\ \hline
      2 & Show the activity level for each hour of every day \\ \hline
      3 & Let you compare hours from multiple days \\ \hline
      4 & Let the user  \\ \hline
      4 & Let the user easily identify hours where more activity can be added \\ \hline
      5 & Let the user identify patients that are active during the night \\ \hline
  \end{tabular}
  \end{center}
\end{table}
 
%//Get an overview of week days and hours in consultation with patientsdrop
