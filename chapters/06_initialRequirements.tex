\chapter{Initial requirements}
\label{ch:initialRequirements}
% I don't even know what to think about this section. Please review and add stuff if you get some ideas. I didn't really participate much in this process. Section should probably be a little longer... HELP ME DEAN HEEEELP ME!
\section{Requirement gathering}
The initial requirements were gathered with the help of a physiotherapist working on a research project at St. Olavs Hospital. Here the activPAL sensor is used to track the activity of patients being rehabilitated after hip-fractures. The only visualization utilized is the one offered by the activPAL software.

Research projects such as the one mentioned above are currently the only arena where physiotherapists use \gls{pi} technology to track the activity level of their patients. We therefore expected the requirements to be changed considerably after the focus group sessions. The initial requirements were created during the first meeting. Questions and issues that arose during the development phase were solved via mail or telephone.

\section{Initial Requirements and User Scenario}
After talking to the domain experts at St. Olavs Hospital two main user scenarios emerged:
\vspace{-3mm}
\begin{enumerate}[itemsep=0cm, parsep=0cm]
  \item Mapping the activity level of patients
  \item In consultation with patients
\end{enumerate}

The current practice of physiotherapists working with the elderly is to first to get an overview of the patients activity level. This information is then used to create a program to improve the activity level of the patient. The activPAL sensor can be used in addition to current techniques to gather more quantitative data to get an even better mapping of the patients level of activity. Scenario 1 is concerned with how to visualize this data so that the physiotherapists quickly can get an idea on how to proceed with the treatment.

After the physiotherapists has an idea of the patients current level of activity the data is presented to the patient. This is usually a discussion about the results of the test and what the patient wants to achieve in terms increased activity. Scenario 2 deals with how the visualizations can be helpful in making the patient understand and becoming aware of the current level of activity.

Using the two user scenarios above we created some simple requirements for the visualizations as a basis before starting development.

\begin{table}[h!]
  \begin{center}
  \begin{tabular}{|c|p{12cm}|}
    \hline
      \textbf{Id} & \textbf{Requirement} \\ \hline
    \multicolumn{2}{|l|}{The visualizations should \ldots} \\ \hline
      IR1 & give an overview of the week \\ \hline
      IR2 & give a summary of the daily activity \\ \hline
      IR3 & show the activity level for each hour of every day \\ \hline
      IR4 & let you compare hours from multiple days \\ \hline
      IR5 & show the activity level for each minute of every day \\ \hline
      IR6 & let the user identify patients that are active during the night \\ \hline
  \end{tabular}
  \end{center}
\end{table}
