\chapter{Initial requirements}

% REVIEW ME, PLEASE!
% REVIEW: I think you might be right about the fact that we need a method chapter about interviews, but should that be placed here or in HCI or another chapter?
\section{Requirement gathering}
The initial requirements were gathered with the help of a domain expert working on a research project at St. Olavs Hospital. Here the activPAL sensor is used to track the activity of patients being rehabilitated after hip-fractures.

Because of the lack of information we had on the current usage of sensor technology in physiotherapy and how this technology works, we conducted an unstructured interview with the domain expert. In the interview we discussed four subjects: how sensor technology like the activePAL is used in physiotherapy and research today, what types of visualizations are used, technical aspects of using the activPAL, and initial requirements. 

Currently there is little or no use of sensor technology in physiotherapy. There are some physiotherapists involved in the research projects using these types of sensors, but they do not use them in their normal practice. 

The only visualization utilized by the domain expert  and physiotherapists using the activPAL sensor are the visualizations that are offered by the software that comes with the sensor, as seen in section~\ref{sec:activPALViz}.

From a technical aspect it is not hard to use the data gathered by activPAL. The data can easily be exported in the common CSV format. CSV is a machine-readable format, which means that can easily be read and parsed by other computer programs. activPAL offers options for exporting both event data (see appendix~\ref{csvDocument}), or raw acceleration data.

After getting an overview, of the current situation and the technical possibilities of the activPAL sensor, we created the initial requirements found in the next section.

\section{Initial Requirements and User Scenario}
After talking to the domain experts at St. Olavs Hospital two main user scenarios emerged:
\vspace{-3mm}
\begin{enumerate}[itemsep=0cm, parsep=0cm]
  \item Mapping the activity level of patients
  \item In consultation with patients
\end{enumerate}

The current practice of physiotherapists working with the elderly is to first to get an overview of the patients activity level. This information is then used to create a program to improve the activity level of the patient. The activPAL sensor can be used in addition to current techniques to gather more quantitative data to get an even better mapping of the patients level of activity. Scenario 1 is concerned with how to visualize this data so that the physiotherapists quickly can get an idea on how to proceed with the treatment.

After the physiotherapists has an idea of the patients current level of activity the data is presented to the patient. This is usually a discussion about the results of the test and what the patient wants to achieve in terms increased activity. Scenario 2 deals with how the visualizations can be helpful in making the patient understand and becoming aware of the current level of activity.

Using the two user scenarios above we created some simple requirements for the visualizations as a basis before starting development.

\begin{table}[h!]
  \begin{center}
  \begin{tabular}{|c|p{12cm}|}
    \hline
      \textbf{Id} & \textbf{Requirement} \\ \hline
    \multicolumn{2}{|l|}{The visualizations should \ldots} \\ \hline
      IR1 & give an overview of the week \\ \hline
      IR2 & give a summary of the daily activity \\ \hline
      IR3 & show the activity level for each hour of every day \\ \hline
      IR4 & let you compare hours from multiple days \\ \hline
      IR5 & show the activity level for each minute of every day \\ \hline
      IR6 & let the user identify patients that are active during the night \\ \hline
  \end{tabular}
  \end{center}
\end{table}
