\chapter{Body-worn sensor technology}

\section{Sensor technology}
Basis technology (mention activPAL) mention gps for tracking of old people

\section{Presentation of sensor data}

\subsection{Nike+}
NikeFuel is a unit of measurement used by all Nike activity trackers. The FuelBand does calculate steps and calories burned, but the NikeFuel is the prime focus of their product line. NikeFuel does not take into account gender, height, weight, but looks purely at activity. Meaning that a mile will award the same amount of points to users of very different physiology. The daily progress (figure \ref{fig:tworings})is represented through a ring that fills up when the FuelBand detects activity, a full ring means that the daily goal has been reached. Progress beyond the daily goal will be displayed in numbers and visual enchantments on the ring if the user surpasses the goal by a certain percentage.

\begin{figure}[h!]
	\centering
		\includegraphics[width=0.9\textwidth]{tworings.png}
		\caption{\footnotesize \textbf{(a):} User halfway to his daily nikefuel goal. \textbf{(b):} User beating his daily limit by 150\% rewarding him with special effects on the ring and a trophy. (Images taken from \cite{fuelbandDcRain} and \cite{fuelbandTechSpce})}.
		\label{fig:tworings}
\end{figure}

The online profile provides detailed information of the users activity, showing steps, calories burned, active time, distance travelled and average fuel. Charts can be displayed for weeks, months or years. This allows the user to track their progress and look at how often they achieve or exceed their goals.\cite{fuelbandTechSpce}.

\begin{figure}[h!]
	\centering
		\includegraphics[width=0.7\textwidth]{week.png}
		\caption{\footnotesize The weekly breakdown visulizations presented by Nike+ to the user. \cite{fuelbandTechSpce}}
		\label{fig:activityBreakdown}
\end{figure}

Virtual trophies are awarded for various achievements such as gathering an X amount of NikeFuel or beating the set goal by a 100\%. These trophies can then be shared with friends or displayed on the public profile to show off achievements. The FuelBand can display simple information such as how far the user is from his daily goal, steps taken, and calories burned. A review has reported that the NikeFuel concept can almost become an addiction and lead to doing some last minute workouts in order to reach the goal \cite{fuelbandDcRain}.

\subsection{Fitbit}
Similar to the Nike+, Fitbit allows the user to set daily goals, but as the Fitbit does not use the NikeFuel system, it enables the user to set 3 separate goals: steps taken, floors climbed, and calories burned. The Fitbit does provide an active score, but there is no emphasis on it. A daily activity breakdown is provided, this breaks the activity levels for the day into 4 categories: sedentary, lightly active, fairly active, and very active. All the goal histories can be viewed on the online profile and can be categorized into day, week, months and years.

\begin{figure}[h!]
	\centering
		\includegraphics[width=0.9\textwidth]{fitbitActivityBreakdown.png}
		\caption{\footnotesize The weekly breakdown visulizations presented by Nike+ to the user. \cite{fuelbandTechSpce}}
		\label{fig:fitbitActivityBreakdown}
\end{figure}

\section{Personal informatics and quantified self}
Currently there are two names that stand out within self-monitoring: Quantified Self, and Personal Informatics. Quantified self is a community of end users who share data and exchange experiences with tools that help them collect information. Personal Informatics on the other hand is academic and developer driven, they desire to understand what makes a good tool, and how the user interacts with such technology.

\subsection{Quantified Self}
In 2011 a movement known as Quantified Self*\cite{quantifiedSelf} had their first conference \cite{bodyHackers}, here people shared data that they had collected about themselves width different types of devices. The goal is to gather as much information about yourself as possible, to learn about yourself through quantitative data. Members of Quantified Self* collect information about everything from sleep patterns and diets to mood and stress levels.

(Cut the next paragraph to shorten this section?)

To promote further development in tools that gather these types of information, the participants of Quantified Self has worked closely with companies and individuals that create personal informatics tools. Devices such as Nike's FuelBand \cite{fuelBand} and Fitbit \cite{fitBit} are results of this cooperation, and both products have been well received.

\subsection{Personal Informatics}
Personal informatics is the label used to classify tools that help people collect personal information for the purpose of self-monitoring and self reflection. These tools are used to help individuals gain self-knowledge about their behaviour, habits, and thoughts\cite{personalInformatics}.

(Cut the next paragraph to shorten the section?)

The Computer-Human Interaction (CHI) conference has since 2010 \cite{chi2010} held workshops and accepted papers on Personal Informatics. The aim is to increase understanding of how the tools affect the user, explore new possibilities, and overall improvement of the user experience.

\section{Medical applications (not a priority)}
Find articles that use body worn sensors for health care