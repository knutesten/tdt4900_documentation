\chapter{Focus Group 1}
\label{ch:focusGroup1}
The primary purpose of the first focus group was to receive feedback on the visualizations we had created, and see if the participants had any ideas of their own. A small part of the focus group was dedicated to understanding what kind of technology and visualizations the participants used in their line of work. This chapter starts of by describing the participants and explains how the focus group session was conducted. The results of the focus group are presented at the end of the chapter.

\section{Participants}
The focus group session consisted of five physiotherapists who were all employed at Trondheim Kommune (municipality of Trondheim) physiotherapy department, but are responsible for different districts within the municipal. We had originally invited six, but one of them could not make it due to a sudden conflict in their schedule. Their work involves visiting patients that are too old or unable to show up at a regular physiotherapist for other reasons. Four of the participants were female and the last one was male, and the ages were primarily between 34 and 36, with one participant being 46. The participants represented a portion of potential users, but no patients were present.


\section{Location and process}
The session was held in a meeting room at St. Olav's Hospital. The participants were seated together around a table facing a live demonstration of the visualizations that was controlled by the assistant moderator. The primary moderator was responsible for conducting the session, providing explanations, and asking questions while not influencing the participants. Video recordings were made of the entire session, so that the participants feedback could be viewed at a later time. The entire session lasted for two hours including a te minute break.

The agenda for the session:
\vspace{-8mm}
\begin{enumerate}[itemsep=0cm, parsep=0cm]
  \item Introductions
  \item Opening questions
  \item Review of visualizations
  \item Short break
  \item Discussion and drawing
  \item Thoughts about session
\end{enumerate}

The introductions was used to learn names and to explain how a focus group works. The participants were informed that they could be critical to the visualizations presented and that the whole idea of a focus group is to get everyones individual opinion, not only the opinion of the group. Though the participants had been informed of project in the invitation, some time was used to explain about the project as well as answer any questions related to the project. The introduction was also used to go through the agenda.

The opening questions were used to get an idea of the participants familiarity with the technology used in this project. This phase was also used to get the participants to talk, to warm them up for the more important review of the visualizations.

After the opening questions the visualizations were presented one by one. In general each visualization was reviewed before the next one was shown, but we did switch between visualizations when this was relevant for comparison or requested by the participants. A few questions were prepared for each visualization to ensure a nice flow in the discussion.

One the review of the visualizations was concluded the participants were allowed to take a short break. During the break we explained about the technical aspects of the project in more detail as well as showing the participants another data set than the one used under the review phase. %Skal vi skrive noe om at vi viste dataen vår som et alternativ og det fikk dem til å åpne mer?

Next, the participants were asked to discuss how they would create the perfect visualizations for their practice. Afterwards the drawings were discussed and reviewed by the other physiotherapists. When this was completed, we discussed some of our findings to clarify if we had interpreted the participants correctly.

The last part of the focus group was used to summarize the session and ask the participants about the experience of participating in a focus group. They were also asked if they would be willing to participate in the next focus group.

\section{Results}
This section contains the results of the first focus group. New sets of scenarios and requirements are presented, and the participant's feedback and comments on the visualizations is covered in some detail. The last section contains a discussion on colour choices and the possibility of printing the visualizations.

\subsection{Scenarios}
The first part of the focus group session was used for discussing how the technology could be used in practice. After going through the video and analysing the discussions, we modified the initial scenarios and created two new scenarios where visualizations would be a helpful tool: 
\begin{table}[!h]
  \centering
  \begin{tabular}{|c|p{10cm}|}
    \hline
    \textbf{Id} & \textbf{Scenario} \\ \hline
    S1-1 & Analysing patients current activity level either individually or in cooperation with other physiotherapists. \\ \hline
    S1-2 & In consultation with the patient. \\ \hline
    S1-3 & In communication with other health care personnel. \\ \hline
    S1-4 & In consultation with next of kin. \\ \hline
  \end{tabular}
  \caption[Scenarios after Focus Group 1]{The scenarios after the first focus group.}
\end{table}

\subsection{Requirements}
Analysing the discussion and feedback on the visualizations, we revised the initial requirements and created a new set of requirements to be used at the second focus group. The new requirements were divided into functional requirements and user experience requirements:

\begin{table}[h!]
  \begin{center}
  \begin{tabular}{|c|p{12cm}|}
    \hline
      \textbf{Id} & \textbf{Requirement} \\ \hline
    \multicolumn{2}{|l|}{The visualizations should \ldots} \\ \hline
      R1-1 & give the user an overview of the week where the days are classified based on national or personal goals \\ \hline
      R1-2 & show the activity level for each hour of the day \\ \hline
      R1-3 & make it simple to identify periods of inactivity \\ \hline
      R1-4 & make it possible to compare multiple days \\ \hline
      R1-5 & make it easy to identify hours of the day where activity can be added \\ \hline
      R1-6 & show the activity level compared to national or personal goals \\ \hline
      R1-7 & let the user identify patients that are active during the night \\ \hline
      R1-8 & let the user compare two separate weeks to see the patients progress \\ \hline
      R1-9 & be printable in grayscale \\ \hline
  \end{tabular}
  \end{center}
  \caption[Functional requirements after the first focus group.]{Functional requirements from the first focus group.}
\end{table}

\begin{table}[h!]
  \begin{center}
  \begin{tabular}{|c|p{12cm}|}
    \hline
      \textbf{Id} & \textbf{Requirement} \\ \hline
    \multicolumn{2}{|l|}{The visualizations should \ldots} \\ \hline
      R1-10 & not be judgemental towards the patients activity level \\ \hline
      R1-11 & be honest about the patients activity level \\ \hline
      R1-12 & motivate the patient to be more active \\ \hline
      R1-13 & be intuitive and easy to understand for the user \\ \hline
      R1-14 & be easy to explain to the patient \\ \hline
  \end{tabular}
  \end{center}
  \caption[User experience requirements after the first focus group.]{User experience requirements from the first focus group.}
\end{table}

\subsection{Visualizations}
A large part of the focus group session was used to review the visualizations from prototype 1. Each visualization was reviewed one at a time, and the participants gave positive and negative feedback. We will now go through the most important parts of the feedback for each visualization group.

\subsubsection{Overview charts}
U1 was generally well received as a good way to get an overview of the week. There was some confusion as to how the days were classified, and it was suggested that one should be able to set custom goals to be used in the classification. One of the participants stated that the use of sad smilies would be judgemental toward the current activity level of the patient. Because a lot of the patients that receive help from physiotherapists have a low level of activity it was a fear that all days would be classified as sad smilies reducing the motivation of the patient. Using colours instead of smilies was suggested.

U2 was not well received. The participants did not like the way the 24 hour blocks were divided into four rows. This made the them look like days in a calendar, which was confusing for participants. U2 contained too much information and it was hard to get a feel for the overall activity of the day because the hours were split on four rows. The participants also had a hard time comparing days in different categories because the day-boxes were too far apart.

In general the participants liked the idea of an overview chart. Classifying the days should be made clearer by adding customizable goals. Smilies should be removed because they can be judgemental toward the current activity level of the patient. The overview charts should not show more detail than classifying the days.

\subsubsection{Aggregated charts}
F1 was seen as easy to understand because of the familiarity of pie charts. Several of the participants did not like the fact that nighttime was added because the patients are supposed to be inactive during the night and the chart gave an unnecessary bad impression of the day as a whole. It was also perceived as hard to separate good and bad days because the percentage of activity always remained very small compared to inactivity. They liked the ability to see the entire week at once.

F2 was not as well liked as F1. Though the participants like the figures that illustrated the different types of activity, they did not like the box approach for visualizing the percentages. The participants thought it was easier to see the distribution using the pie chart compared to the box chart. One of the participants suggested adding the illustrations to the pie chart. Also here the participants wanted to remove nighttime.

The participants were positive to F3, the ball chart. They liked the fact that the visualizations showed the length of the intervals. Some participants wanted nighttime to be removed, while others felt that it was interesting to see if a patient walked during the night. The participants agreed that this type of visualization is too complex to be show to patients, however it could be shown to other health care personnel. The highlighting functionality was perceived as redundant, since it was already easy to identify the largest periods of inactivity.

The participants felt the F2, box chart, was hard to understand. They liked the illustrations of the F2 and wanted to add them to F1, the pie chart. There should be an option to see the entire week simultaneously. F3 was a good way to get an overview of the length of activity intervals. Highlighting was not needed for F3.

\subsubsection{Timeline and Clock Charts}
T1, blocked timeline, was very well received by the participants, especially in week view. The participants praised the ability to easily see the patients habits. It was also stated that nighttime no longer got a negative impact on the visualization because it could easily identified and ignored. One of the participants liked to have the ability to see if the patient was awake during the night (in the example data one can see the 10 minutes of activity at 4 AM). The participants stated that it was very easy to get a quick and detailed overview of the entire week, and it was easy to identify hours where more activity could be added. The participants suggested using different type of colours to make the gradient clearer.

T2 was not well received. The participants felt that the red colour, representing inactivity, was way to dominating. The periods of activity were hidden by all the red. They also stated that it was hard to see periods of walking activity, because they were hidden by all the green and yellow (standing activity). When asked about the need for more detail, the participants answered that it was not useful with greater detail than the 24 hour blocks provided in T1.

T3 displayed two clocks instead of a timeline. The participants did not like this visualization. They felt it was hard to identify when different events were occurring. They also stated that it was hard to compare multiple days because the circles could not be placed directly below each other like the timelines.

T4 was better received than T3. Some of the participants felt it was confusing that the clock had 24 hours. It was also expressed concern that the red portions representing inactivity stood out too much, making it hard to see the periods of activity. Also it was seen as hard to compare multiple days.

The participants liked T1 because it was easy to get an overview for both the day and week. The participants felt that there was no point in providing greater detail then 24 hour blocks. T2, T3 and T4 was seen as too hard to read, both because it was too detailed and because the active periods were buried by periods of inactivity.

\subsection{Colours and Printouts}
The participants were critical to some of the colour choices used in the visualizations. Several of the participants were critical to the use of red to represent a sedentary position, because the colour became too dominant which might demotivate patients. Several participants also complained that it was hard to see the gradient colours in T1. When the participants were shown the same visualization on a laptop instead of the projector, they saw the gradient much better. For visualizations utilizing gradients it is important to test the screen quality before they are used. Otherwise the diagram might be misinterpreted. 

One of the participants asked if the visualizations could be printed out, as they currently are not provided with portable laptops or tablets. Though the visualizations can be printed, they were not designed to be used other than on a computer. The participants also informed that they do not have access to colour printers. This means that the visualizations also should have grayscale versions for printing. 
