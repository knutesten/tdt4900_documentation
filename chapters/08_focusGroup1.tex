\chapter{Focus group 1}


\section{Procedure}
This section gives information

\subsection{Participants}
The focus group session consisted of 5 participants who are all employed at the Trondheim municipals physiotherapy department, but were responsible for different districts within the municipal. Four of the participants were female and the last one was male, and the ages were primarily between 34 and 36, with one participant being 50. All of the participants were positive technology and used their smartphones and laptops often in their leisure time. However they were not accustomed to using tables, technology or visualizations when talking to patients or in the field. %Ikke sikker på at man skal ta med den siste linjen her. 
The participants represented a selection of potential users, but no patients were present. A moderator was present 

\subsection{Location and plan}
The session was held in a meeting room at St. Olav's Hospital. The participants were seated together around a table facing a live demonstration of the visualizations that was controlled by the assistant moderator. The primary moderator was responsible for conducting the session, providing explanations, and asking questions while not influencing the participants.

%I don't like the part very much, but not sure what to do
We started with introductions, explaining our work and a general outline for the session. The session itself consisted of three parts: Opening questions, visulizations, open discussion. Opening questions consisted of us getting some information about their work, the current visual aids they use in their work, and what they are interested in using visualizations if presented with the opportunity. The second and biggest part of the session consisted of displaying the visualizations and receiving feedback on what can be improved removed or added. An open discussion where participants were able to freely share, draw which were eventually discussed discuss. At the end we summarized our initial interpretation of the answers that had been given during the session to confirm that they were correct.

\section
\section{Results}
Scenarios:
Use for self or other physiotherapists
In communication with other health care personnel (hjemmetjenesten)
In consultation with the patient
In consultation with next of kin 

\section{Interpretation}

