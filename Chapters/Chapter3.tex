\chapter{Related Products and Research} % Main chapter title

\label{Chapter3} % For referencing the chapter elsewhere, use \ref{Chapter1} 

\lhead{Chapter 3. \emph{Related Product and Research}} % This is for the header on each page - perhaps a shortened title

\section{Related Products}

A movement known as Quantified Self*\cite{quantifiedSelf} who had their first conference in 2011\cite{bodyHackers}, is interested in gaining more self knowledge through self-tracking. This goes far beyond simply tracking physical activity, they wish to quantify as much data as possible such as mood, sleep patterns, and diet. The exchange of information within the movement between users and tool makers allows for new, exciting and relevant technology to be developed. The FuelBand\cite{fuelBand}, Fitbit\cite{fitBit} are some of results of this cooperation, and both products seem to be well received by the movement. %ADD CITATION, ARTICLE NOT DONE YET. CHECK BACK SOON

\subsection{Wrist worn activity monitors}
The Nike+ Fuelband\cite{fuelBand} was the first wrist worn activity monitor to be released. The wristband contains a 3-axis accelerometer that tracks the users wrist movement, and activity awards the user with \emph{Nikefuel}. Nikefuel\cite{nikefuel} is a unit of measurment used by all Nike activity trackers, however there are no details on how activity is converted to Nikefuel as the algorithm is proprietary. The wristband does calculate steps and calories burned, but the Nikefuel is the prime focus of their productline. Nikefuel does not take into account gender, height, weight but looks purely at activity, 

Users can set a daily Nikefuel goal, share it with others, and compare themselves to friends and professional athletes. 

\section{Related Research}

