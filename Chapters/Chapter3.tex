\chapter{Related Products and Research} % Main chapter title

\label{Chapter3} % For referencing the chapter elsewhere, use \ref{Chapter1} 

\lhead{Chapter 3. \emph{Related Product and Research}} % This is for the header on each page - perhaps a shortened title

%We should write about the many smartphone apps that offer self-tracking.
\section{Related Products}
In 2011 a movement known as Quantified Self*\cite{quantfiedSelf} had their first conference \cite{bodyHackers}, here people shared data that they had collected about themselves width different types of devices. The goal is to gather as much information about yourself as possible, the goal being learn about yourself through quantitative data. Members of Quantified Self* collects information about everything form sleep patterns and diets to mood and stress levels.
%Maybe make it clearer that it is an exchange of information between QS and developers?
The exchange of information within the movement between users and tool makers allows for new, exciting and relevant technology to be developed. The FuelBand\cite{fuelBand}, Fitbit\cite{fitBit} are some of results of this cooperation, and both products seem to be well received by the movement. %ADD CITATION, ARTICLE NOT DONE YET. CHECK BACK SOON

\subsection{Nike+}
The Nike+ concept has existed since 2010 and is the brand name Nike uses for sports related activity tracking. It started with a sensor in the shoe and a receiver connected to an iPod. Since then the Nike+ line has expanded into Kinect Games, sport watches and shoe implants \cite{nikeProducts}. As the product line increased so did the community around it. Every Nike+ device now requires an online profile where user can store information, create goals, and look at their history. However the Nike+ Fuelband \cite{fuelBand} is the first product aimed at monitoring the user outside of their workout, and motivating them to reach the daily activity goal

The Fuelband\cite{fuelBand} was released in early 2012 and was the first commercial wrist worn activity monitor. The wristband contains a 3-axis accelerometer that tracks the users wrist movement. Recorded activity is converted to \emph{Nikefuel}, Nikefuel\cite{nikefuel} is a unit of measurement used by all Nike activity trackers, however there are no details on how activity is converted to Nikefuel, as the algorithm is proprietary. The wristband does calculate steps and calories burned, but the Nikefuel is the prime focus of their product line. Nikefuel does not take into account gender, height, weight, but looks purely at activity. Meaning that a mile will award the same amount of points to users of very different physiology.

The information from the fuelband can be uploaded to an iPhone or laptop which then send the information to the Nike+ servers and update the users profile. The online profile provides detailed information of the users activity, showing steps, calories burned, active time, distance travelled and average fuel. Charts can be displayed for weeks, months or years. This allows the user to track their progress and look at how often they achieve or go way above their goals.\cite{fuelbandTechSpce} 

Virtual trophies are awarded for various achievements such as gathering an X amount of NikeFuel or beating your set goal by a 100\%. These trophies can then be shared with friends or displayed on the public profile to show off your achievements. The band itself can be used to view simple information such as how far you are from your daily goal, steps taken, and calories burned. A review has reported that the NikeFuel concept can almost become an addiction and lead to doing some last minute workouts the goal in order to achieve that satisfying sensation \cite{fuelbandDcRain}
%I agree that this might be a little too specific, since we are not using this technology.
%Unsure if specific technical details should be mentioned, such as that it communicates through bluetooth etc. I figure the relevancy of these products is more on how they do things and what we can learn.

\section{Related Research}
%I would like to kill myself.
During the CHI 2013 Workshop \cite{chi2013} one of the accepted papers discussed different approaches to visualize data from personal informatics devices. %cite here
An application called Spark was created to visualize step-count data form a FitBit trakcer. %cite here
Three different visualizations were created, based on abstract art from artists like Wassily Kandinsky and Piet Mondarian: Spiral, flora and bucket. %Should we describe the visualizations more in detail?
Though the three graphs are nice to look at, at least in the writers opinion, an evaluation of the usability and motivating factor of the visualizations in Spark is sill ongoing. 
