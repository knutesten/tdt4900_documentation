\chapter{Prototype} % Main chapter title

\label{Chapter5} % For referencing the chapter elsewhere, use \ref{Chapter1} 

\lhead{Chapter 5. \emph{Prototype}} % This is for the header on each page - perhaps a shortened title}

\section{Initial Graphs}

%We must settle on a name for these types of graphs
%Changed this section, have a look.
\subsection{Fractional charts}
These types of charts shows the sum of time spent sedentary, standing, and walking. Summing over the three classifications makes it easy to get an overview of the day as a whole. Though the summation gives a great overview, details are lost and it is not possible to pinpoint when each activity occurred during the day. (Name of chart)* can be useful for quickly identifying days with a low amount of activity, before spending time looking at more detailed visualisations. It might also serve as a wakeup call for people with low activity level.

\subsubsection{Pie/box chart}
%Don't think we need to show a picture of our sketch for a pie chart.
The first draft for a pie chart can be seen in figure \ref{fig:scientificPie}, is what we primarily think of when the word pie chart is mentioned. We have discussed if we should display percentage, time or both and how this should be implemented. Putting them next to legends, on the pie chart itself, or having it appear when a section of the pie chart is selected are all possible solutions.

%Would move this paragraph to another section. Maybe one about experimental charts or something. This does not have that much to do with a pie chart?
Instead of using solid colours the pie chart a set of circles or "balls" can be used. Each ball represents a interval of continuous activity. So a small number of large sedentary balls represent few but long periods in a sedentary position. A ball chart visualization has more elements to take in, but help illustrate the subjects activity pattern.


\begin{figure}[h!]
	\centering
		\includegraphics[width=0.5\textwidth]{placeholder.jpg}
		\caption{\footnotesize A sketch of how we envision the scientific pie chart}
		\label{fig:scientificPie}
\end{figure}

\subsubsection{Symbolic}
%Should we refer to our advisor here?
Our Initial approaches were very scientific, therefore our advisor suggested we think a bit more symbolic and artistic. One of the suggestions (Figure\ref{fig:symbolicPie}) was using stick figures to represent an activity. The size of each stick figure would reflect time spent in each activity, i.e. a large walking figure would represent a high amount of hours walking. %SKAL VI HA TELL HE ELLER IKKE, SI NOE OM HVORFOR/HVORFOR IKKE
%We can leave this section here as long as we change the section name (not the subsubsection) to reflect that we are talking about graphs that sum over the different activities. That is what these graphs have in common.
\begin{figure}[h!]
	\centering
		\includegraphics[width=0.5\textwidth]{placeholder.jpg}
		\caption{\footnotesize Rough drafts of a symbolic "pie chart"}
		\label{fig:symbolicPie}
\end{figure}

\subsection{Timeline}
Timeline visualisations are effective at illustrating when various activities occurred during the day. Such a bar can be used to identify points during the day where the subject is in a sedentary position for too long.

\subsubsection{Scientific}
A simple yet effective timeline was our first approach. A sketch can be seen in figure \ref{fig:simpleTimeline}. The entire day is represented by a horizontal bar, where each colour corresponds to an activity. A symbol can be used to highlight where sedentary time is higher than a set limit. The size of the bar would have to be quite big in order for a user to inspect some of the shorter time frames, therefore the possibility for zoom functionality should be explored.

%Changed a bit here, review.
Instead of having a continuous scale, a blocked approach can be used. The timeline would be divided into 24 blocks, each block corresponding to an hour. A gradient colour scale would be used to represent the amount of activity inside the hour block. This should make it easy to identify hours in the day where prolonged sedentary positions are present. Giving feedback about specific hours might make it easier to interpret and make use of the chart, because you are alerted to certain hours of the day where you should be more active.

\begin{figure}[h!]
	\centering
		\includegraphics[width=0.5\textwidth]{placeholder.jpg}
		\caption{\footnotesize Rough drafts of a symbolic "pie chart"}
		\label{fig:simpleTimeline}
\end{figure}

\subsection{Symbolic}
%Don't think you should use the past tense here, I changed it have a look.
An alternative approach to the standard timeline is to use clocks to represent the day as shown in figure \ref{fig:clock}. Two clocks are used to represent the day, with the first clock going from 1 to 12 and the second from 13 to 24.
%Wasn't it the background that should have a darker background? I guess this would work too though.
 In addition a darker colour palette could be used during hours of the night. This would deter focus away from these hours and help indicate that a high sedentary time is expected in the time period.

\begin{figure}[h!]
	\centering
		\includegraphics[width=0.5\textwidth]{placeholder.jpg}
		\caption{\footnotesize Rough drafts of a symbolic "pie chart"}
		\label{fig:clock}
\end{figure}
%Maybe a little informal? 
To really illustrate the day we came up with the idea to animate the day by using stick figures. A timeline will be drawn in real time while stick figures simultaneously perform the activities depicted on the timeline. By displaying the day gradually we hope that the subject will gain a firm understanding of their day. This means that this visualization can not be used to gain a quick overview, but is intended to be used when viewing a day for the first time.

The more motivational approach would be to replace the stick figure with an analogy or metaphor. Instead of a stick figure, a flower could be used. Activity would allowed the plant to get sunlight, making it grow. Sedentary positions would make the weather cloudy and the flower would be unaffected.

\section{Protoype goal/focus}

\section{Prototype implementation}

\section{Findings}
