% Chapter 1

\chapter{Introduction} % Main chapter title

\label{Chapter1} % For referencing the chapter elsewhere, use \ref{Chapter1} 

\lhead{Chapter 1. \emph{Introduction}} % This is for the header on each page - perhaps a shortened title

\section{Purpose/Motivation}
The elderly population is in a constant rise, currently 15\% of the Norwegian population is above the age of 65. By the turn of the century the elderly population is expected to double \cite{elder}. The initial rise of the elder population will be most noticeable in those below the age of 80, as seen in figure \ref{fig:elderPopulation}. After 2025 a great increase in the population above the age of 80 is expected. The Norwegian Institute of Public Health* reports that two out of three above the age of 75 consider themselves having "good health" but only a third preserve this level of health until death \cite{elder}.

\begin{figure}[h!]
	\centering
		\label{fig:elderPopulation}
		\includegraphics[width=0.5\textwidth]{eldrevekst.png}
		\caption{\footnotesize Elder population estimation \cite{elder}}
\end{figure}
The Lancet published one of the largest ever systematic efforts to describe the global health situation was conducted in 2010. One of their many findings was that since 1970 men and women have gained an additional ten years to their life expectancy, but spend more time living with injuries and illness \cite{globalBurden}. %KOMMER TILBAKE HIT

%Make the ratio thing more clear, there was confusion
In the late 1990s the Norwegian statistical bureau published a paper predicting a major increase in the elderly population. The working population compared to the retired currently holds a ratio of 5 to 1. This is expected to drop below 3 by the year 2040. Because of this the greatest challenges within wellfare are expected to occur between 1998 and 2020 \cite{eldreEksplosjon}. 

With such predictions the Norwegian government formed a committee (referred to as Hagen-utvalget*) that would investigate the current situation and suggest solutions for accommodating the increase in the percentage of elderly \cite{haagen}. One of the conclusions in the report was that too little of today's technology is incorporated as welfare technology* for the elderly. A Danish report refereed to by Hagen-utvalget states that around 20\% of the tasks performed by health care personnel could be completely or partly replaced by technology \cite{kmd}. 

To handle the rising percentage of elderly, Hagen-utvalget suggest a national three step program that focuses on using welfare technology to diminish falls, social isolation and cognitive failure. Thus improving the overall quality of life for the elder population as well as reducing the workload for health care personnel. 
Step 3 states that:
\begin{quote}
\textit{``Opt on technology that stimulates, activates and structures daily life.''}
\end{quote}


%problem for elderly is that they dont walk
%Not walking is bad
%Can getting elderly to realize how little they walk throug fancy graphs make them walk more.. (dont know if we can an)
\section{Project context/Thesis Scope}
This project is a cooperation between the Department of Computer and Information Science (IDI)* at the Norwegian University of Science and Technology (NTNU) and St. Olavs University Hospital in Trondheim. The project...

\section{Research questions}

\section{Research method}

\section{Thesis outline}
