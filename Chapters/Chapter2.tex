\chapter{Technology} % Main chapter title

\label{Chapter2} % For referencing the chapter elsewhere, use \ref{Chapter1} 

\lhead{Chapter 2. \emph{Technology}} % This is for the header on each page - perhaps a shortened title

\section{HTML5}
HTML is a markup language for the creation of web pages. HTML describes the structure and the contents of the web page. In later years, the need for advanced styling and complex interaction with web pages has made CSS and JavaScript increasingly popular. HTML5 was created as a response to this, HTML5 is an umbrella term for creating web pages using HTML5, CSS3 and JavaScript.

HTML5 has simplified the syntax compared to earlier versions. New tags have been added to better represent the modern web page elements. Other features include media tags which greatly simplifies adding multimedia content, such as playing audio and video files. More importantly for our project is the extensive support for interactive and animated graphics through the \emph{canvas}- and \emph{svg}-tag.

The new features of HTML5 and CSS3 make it much easier to create web applications for multiple platforms and screen sizes. After the smartphone and tablet revolution creating responsive and adaptable websites has become more important. The new features included in HTML5 give large amount of flexibility with respect to the user interface and graphical visualizations.

%Is this section too short? On the other hand, is there anything more to write about CSS3 in this context?
\section{CSS3}
\emph{Cascading Style Sheets} (CSS) is a language used to describe the styling of an HTML document. Using CSS the size, color and look of HTML elements can be configured. A new feature in CSS3, which is part of HTML5, is \emph{Media Queries}. With Media Queries it is possible to specify different styling relative to the size of the screen. This functionality is useful when creating applications that target devices with different screen sizes, such as smartphones, tablets and laptops. 

%Is this section to short? Is it too stripped?
\section{JavaScript}
JavaScript is the main scripting language for web pages. It is a client-side scripting language that allows programmers to add functionality to otherwise static HTML-pages. While CSS3 takes care of the styling of HTML-elements, JavaScript is used to create customized behaviour. All modern browsers have JavaScript engines/interpreters that compile and run JavaScript code.

JavaScript is now an industry standard maintained by ECMA International. The standardized version of the script is named ECMAScript. Today ECMAScript and JavaScript are used interchangeably, and JavaScript is often used to refer to ECMAScript. Because different browsers have different implementations of the JavaScript engine, slight variations in the way JavaScript code will run on these browsers exists.

Together with HTML5 and CSS3, JavaScript is great for creating web applications that can be designed to run on both mobile and stationary devices. JavaScript has a multitude of useful open source libraries that can be used to create complex user interaction, animation, and custom graphics.

\section{Data-Driven Documents}
One of the challenges in this project is to create different visualizations to represent the activity patterns of subject. Creating custom graphics in HTML5 can be done using both the canvas- and the svg-tag. In this project SVG is used. Scalable Vector Graphics (SVG) is an image format that uses XML encoding. All popular browsers, and most mobile devices, support rendering of the svg-tags.

Creating graphics using svg-tags directly is cumbersome and time consuming. Therefore a framework that greatly simplifies this task is used. \emph{Data-Driven Documents} (D3) is an open source framework written in JavaScript. D3 supports animation and interaction with the SVG elements. This allows us to explore if animation and interactivity can improve the readability of the visualizations.

\section{activPAL}
To record the activity pattern of elderly a device called \emph{activPAL} will be used. activPAL has the shape of a small rectangle and is worn on the thigh. When the device is active it continuously records accelerometer data using an internal accelerometer. This data can be interpreted using a algorithms provided by PAL Technologies.

When the activity data has been gathered the \emph{Intelligent Activity Classification}-algorithm, provided by PAL Technologies, is used to classify the data into three different types of behaviour: sitting/laying, standing and walking. Because activPAL is worn on the thigh the accelerometer is unable to detect the difference between sitting and lying. Number of steps is also counted when walking. This data can be used to calculate how fast the person was walking.

%Is this section too short and a little strange? Maybe it should be moved to the end of the section?
Several studies have concluded that the activPAL is viable for recording and classifying activity \cite{grant2006, ryan2006, grant2008, tsavourelou}. activPAL has also been used in multiple studies for obtaining and analysing activity patterns \cite{grant2010, lord, ryan2010}.

Though this projects main focus is how to present the data and not analysis of it, the accuracy is not a huge concern. It is however important that the data presented to test subjects* reflects a realistic activity pattern. Another important factor is to show that data from a simple monitoring device can be used to create complex visualizations.
