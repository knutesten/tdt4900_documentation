\chapter{Introduction} % Main chapter title

\label{Chapter2} % For referencing the chapter elsewhere, use \ref{Chapter1} 

\lhead{Chapter 2. \emph{Technology}} % This is for the header on each page - perhaps a shortened title

\section{HTML5}
HTML is a markup language for crating web pages. HTML contains the structure and the contents of the web page. In later years, the need for advanced styling and complex interaction with web pages has made CSS and JavaScript increasingly popular. HTML5 was created as a response to this, HTML5 is an umbrella term for creating web pages unsing HTML5, CSS3 and JavaScript.

HTML5 has simplified the syntax of the previous versions, making it easier to read. New tags have been added to better represent the modern web page. Other features include media tags wich greatly reduces the dificulty of adding multimedia content, such as playing audio and video files. More importantly for our project is the extensive support for interactive and animated graphics through the \emph{canvas}- and \emph{svg}-tag.

The new features of HTML5 and CSS3 makes it much easier to create web applications for multiple platforms and screen sizes. After the revolution of the smartphone and tablet creating responsive web sites that change according to screen size and platform has become increasingly important. The new features of HTML5 and CSS3 gives a lot of flexablity and the ability to construct an application that can reach a broad amount of devices.

\section{CSS3}
\emph{Cascading Style Sheets} (CSS) is a language used to describe the styling of an HTML document. Using CSS the size, color and look of different HTML elements can be configured. A new feature in CSS3, which is part of HTML5, is \emph{Media Queries}. With Media Queries it is possible to specify different styling relative to the size of the screen. This functionality is usefull when creating applications that target devices with different screen sizes, such as smartphones, tablets and laptops. 

\section{JavaScript}
JavaScript is the main scripting language for web pages. It is a client-side scripting language that allows programmers to add functionality to otherwise static HTML-pages. While CSS3 takes care of the styling of HTML-elements, JavaScript is used to create customized behaviour. The JavaScript code is compiled and run by the client using the web browser. All modern browsers support JavaScript.

JavaScript is now an industry standard maintained by ECMA international. The standarized version of the script is named ECMAScript. Today ECMAScript and JavaScript are used interchangeably, and JavaScript is often used to refere to ECMAScript. Because different browsers have different implementation of the JavaScript engine, slight variations in the way JavaScript code will run on these browsers exists. For programmers unfamiliar with such variations debugging JavaScript code can be a frustrating and time consuming task.
        
\section{Data-Driven Documents}
\emph{Data-Driven Documents} (D3) is an open source framework written in JavaScript. The framework is highly flexible and can be used to create a multitude of visualizations. D3 is based on standard web technology, HTML, CSS and SVG, and is supported by all the major Internet browsers.

\section{activPAL}
Recording the activity pattern of the test persons will be done using a device called activPAL \ref{fig:activePAL}. activPAL is a small device that is taped to the test persons thigh. While active the device continuously records acceleromter data in three axis. The accelorometer data are used to classify the periods when the test person was sitting, standing and walking.

When the data has been gathered the device can be connected to a computer running software delivered by PAL Technologies. The software contains proprietary algirithms, \emph{Intelligent Acitivty Classification}, to classify the test persons activites into sitting, standing and walking.

%Maybe write something about the configuration that is needed to spedify what is considered walking etc.

After running the classification algorithms the data can be exported as a csv-file (comma-separated values). The csv-file file is then parsed, using JavaScript, to created input for the different graphs.
