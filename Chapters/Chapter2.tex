\chapter{Introduction} % Main chapter title

\label{Chapter2} % For referencing the chapter elsewhere, use \ref{Chapter1} 

\lhead{Chapter 2. \emph{Technology}} % This is for the header on each page - perhaps a shortened title

\section{HTML5}


\section{JavaScript}
JavaScript is the main scripting language for web pages. It is a client-side scripting language that allows programmers to add functionality to otherwise static HTML-pages. The JavaScript code is compiled and run by the client using the web browser. All modern browsers support JavaScript, though some newer functionality may differ between the different browsers.

JavaScript is now an industry standard maintained by ECMA international. The standarized version of the script is named ECMAScript. Though JavaScript is mostly compatible with the newest version of ECMAScript, it does contain some additional functionality. Because different browsers use different dialects and interpreters of of the same ECMAScript standard, a web application running correctly on one browser may not run correctly on other browsers. Often making debugging difficult and time consuming for programmers unfamiliar with the differences between the browsers. 
        
\section{Data-Driven Documents}
\emph{Data-Driven Documents} (D3) is an open source framework written in JavaScript. The framework is highly flexible and can be used to create a multitude of visualizations. D3 is based on standard web technology, HTML, CSS and SVG, and is supported by all the major Internet browsers.

\section{activPAL}
Recording the activity pattern of the test persons will be done using a device called activPAL \ref{fig:activePAL}. activPAL is a small device that is taped to the test persons thigh. While active the device continuously records acceleromter data in three axis. The accelorometer data are used to classify the periods when the test person was sitting, standing and walking.

When the data has been gathered the device can be connected to a computer running software delivered by PAL Technologies. The software contains proprietary algirithms, \emph{Intelligent Acitivty Classification}, to classify the test persons activites into sitting, standing and walking.

%Maybe write something about the configuration that is needed to spedify what is considered walking etc.

After running the classification algorithms the data can be exported as a csv-file (comma-separated values). The csv-file file is then parsed, using JavaScript, to created input for the different graphs.
