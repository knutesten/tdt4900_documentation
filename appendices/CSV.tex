\chapter{CSV Document}
\label{csvDocument}

\begin{figure}[h!]
	\centering
		\includegraphics[width=1\textwidth]{csv.png}
		\caption{\footnotesize The CSV document as seen in Libre Calculator}
		\label{fig:csvExtract}
\end{figure}

\begin{table}[h!]
  \begin{center}
  \begin{tabular}{|l|p{10cm}|}
    \hline
    \textbf{Row Title} & \textbf{Description} \\ \hline
    Time & Timestap for when the state started. \\ \hline
    Datacount & The amount of sensor readings the event interval is based on. \\ \hline
    Interval & Duration for the event interval presented in seconds. \\ \hline
    ActivityCode & Represents the activity of the user either sedentary (0), standing (1), or stepping(2). \\ \hline
    CumulativeStepCount & The total amount of steps taken since movement tracking was started. \\ \hline
    (Not Relevant)Activityscore & This row was not needed in our work so we did not investigate the meaning behind it. \\ \hline
   	(Not Relevant)Abs & This row was not needed in our work so we did not investigate the meaning behind it. \\ \hline
  \end{tabular}
  \end{center}
  \caption{A short descriptoin of the rows in Figure~\ref{fig:csvExtract}.}
  \label{tab:csvDescription}
\end{table}