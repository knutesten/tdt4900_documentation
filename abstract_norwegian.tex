\renewcommand{\abstractname}{Sammendrag}
\begin{abstract}
\noindent
Nyere studier har vist at bare én av fem voksne og eldre når den nasjonale anbefalingen om 30 minutters daglig aktivitet. Å øke aktiviteten hos de eldre var en av de viktigste utfordringene for Hagen Utvalget, en utvalg oppnevnt av regjeringen for å løse fremtidige utfordringer i omsorgstjenesten. Hagen Utvalgets rapport trekker frem viktigheten av å bruke ny teknologi for å løse disse utfordringene. Ved hjelp av en sensor kalt activPAL kan vi klassifisere pasienters aktivitet i tre ulike kategorier: gående, stående og stillesittende. Data samlet av sensoren brukes til å lage visualiseringer som illustrerer pasienters aktivitet gjennom en uke. Spørsmålet vi prøver å besvare er: Hvilke visualiseringer er best for å hjelpe fysioterapeuter med å tolke og forstå akselerometerdata fra pasienter, i kommunikasjon med pasienter og andre helsearbeidere? Det ble laget en prototype som ble vurdert i to fokusgrupper bestående av fysioterapeuter. Prosessen var iterativ og tilbakemeldinger fra den første fokus gruppen ble brukt til å modifisere og forbedre prototypen før andre focus gruppen. I tillegg til prototypen, ble scenarier for bruken av systemet og et sett av funksjonelle og brukeropplevelse krav opprettet. Kravene og prototypen danner anbefalinger for hvordan å lage visualiseringer som kan benyttes av fysioterapeuter til å løse spesifikke oppgaver. Alle deltakerne av fokus gruppene var positive til prototypen som ble presentert og kunne se for seg å bruke slik teknologi i sitt eget arbeid. Deltakerne var også overbevist om at bruk av slik teknologi vil øke kvaliteten og effektiviteten på deres arbeid.
\end{abstract}
