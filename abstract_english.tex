\pagestyle{empty}
\begin{abstract}
  \noindent
  \begin{description}
    \item[Background] Recent studies have shown that only 1 in 5 Norwegian adults and elderly reach the national goal of 30 minutes of activity each day. Increasing the activity of the elderly is one of the main foci of Hagen Utvalget, a report requested by the Norwegian government. The report also emphasizes on the use of technology to help solve such health problems.
    \item[Problem] Using a sensor called the activPAL we are able to classify a patients activity into periods spent walking, standing and sitting/laying. The data gathered by this sensor is used to create visualizations illustrating the patients activity throughout a week. The question we aim to answer is: Which visualizations are most fitting to aid physiotherapists in interpreting and understanding \gls{imu} data in communication with patients and other healthcare workers?
    \item[Method] A prototype was created and reviewed in two focus groups with physiotherapists working for Trondheim Kommune. The process was iterative and feedback from the first focus group was used to modify and improve the prototype before the second focus group.
    \item[Results] Use cases, functional and user experience requirements were created for a system using sensors such as the activPAL to monitor activity of patients. A prototype was created containing 5 visualizations which satisfies most of the requirements gathered during the two focus groups. 
    \item[Conclusion] All the participants of the focus groups were positive to using system such as the one created in this project in their own work. The participants were also convinced that using such technology would improve the quality and effectiveness of their work.
  \end{description}
\end{abstract}
