\pagestyle{empty}
\begin{abstract}
  \noindent
Recent studies have shown that only 1 in 5 Norwegian adults and elderly reach the national goal of 30 minutes of activity each day. Increasing the activity of the elderly is one of the main foci of Hagen Utvalget, a committee appointed by the Norwegian government to solve future challenges in care service. The report emphasizes on using technology to help solve such health problems. Using a sensor called the activPAL we are able to classify a patients activity into periods spent walking, standing and sitting/laying. Data gathered by the sensor is used to create visualizations illustrating the patients activity throughout a week. The question we aim to answer is: Which visualizations are most fitting to aid physiotherapists in interpreting and understanding accelerometer data in communication with patients and other healthcare workers? A prototype was created and reviewed in two focus groups with physiotherapists working for Trondheim Kommune. The process was iterative and feedback from the first focus group was used to modify and improve the prototype before the second focus group. In addition to the prototype, scenarios for the use of the system and a set of functional and user experience requirements were created. All of the participants were positive to the prototype presented and could see themselves using such a system in their work. The participants were also convinced that using such technology would improve the quality and effectiveness of their work.
\end{abstract}
