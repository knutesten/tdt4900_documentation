\chapter{Methodology} % Main chapter title

\label{chapter4} % For referencing the chapter elsewhere, use \ref{Chapter1} 

\section{Background}
All research methods fall into two categories: Quantitative and qualitative research \cite{realWorldResearch}. Qualitative research is suitable when one wishes to gain an understanding of an underlying phenomenon through explanations and reactions the subjects bring forward. To achieve this the researches have to interview and observe the subjects who are evaluating the object. Quantitative research are appropriate when one wishes to gain a broader understanding of a phenomenon by seeing the effects of some manipulation or activity on the phenomenon. Unlike qualitative research, quantitative research focuses on measurement, quantification and generalization of data. This in turn promotes comparison and statistical analysis of the data gathered.



\section{Focus group}
Focus group, or group interview is an informal technique that can help software developers identify the users needs and feelings about the system in question. The technique can be used both before interface design and long after implementation. 

According to Nielsen \cite{focusGroup} the focus group should have at least six participant to maintain a flowing discussion and provide different perspectives. The participants should be representative of the intended users, or be the final users themselves. Sessions normally last two hours and are run by a moderator. The moderator is responsible for keeping the discussion relevant to the session goals while letting everyone contribute and not hindering the input of ideas and comments from the participants. %Liker ikke helt den siste setningen
A single session may not be representative enough or can get sidetracked, it is therefore important to run more than one focus group.

Nielsen \cite{focusGroup} discusses two pitfalls with focus groups: Sessions are in groups therefore the participants are unable to test the system themselves, so a demo is normally provided instead. The problem with such a demo is that the participants never have to question what to do next or consider the meaning of the screen options. The second problem is that what participants say they want is not necessarily what they need, or things described by the moderator might be perceived differently by the participants. By providing concrete examples through prototypes of the technology, one can minimize this problem.
